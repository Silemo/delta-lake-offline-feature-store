%%%%%%%%%%%%%%%%%%%%%%%%%%%%%%%%%%%%%%%%%%%%%%%%%%%%%%%%%%%%%%%%%%%%%%%%%%%%%%%%%%%%%%%%
%%%%%%%%%%%%%%%%%%%%%%%%%%%%%%%%%%%%%%%%%%%%%%%%%%%%%%%%%%%%%%%%%%%%%%%%%%%%%%%%%%%%%%%%
%%                                                                                    %%
%%               MASTER THESIS - GIOVANNI MANFREDI - 2024 - KTH                       %%
%%                                                                                    %%
%%%%%%%%%%%%%%%%%%%%%%%%%%%%%%%%%%%%%%%%%%%%%%%%%%%%%%%%%%%%%%%%%%%%%%%%%%%%%%%%%%%%%%%%
%%%%%%%%%%%%%%%%%%%%%%%%%%%%%%%%%%%%%%%%%%%%%%%%%%%%%%%%%%%%%%%%%%%%%%%%%%%%%%%%%%%%%%%%
%%
%%%%%%%%%%%%%%%%%%%%%%%%%%%%%%%%%%%%%%%%%%%%%%%%%%%%%%%%%%%%%%%%%%%%%%%%%%%%%%%%%%%%%%%%
%%                                PREAMBLE
%%%%%%%%%%%%%%%%%%%%%%%%%%%%%%%%%%%%%%%%%%%%%%%%%%%%%%%%%%%%%%%%%%%%%%%%%%%%%%%%%%%%%%%%
%%
%% INFORMATION ON THIS PREAMBLE:
%%
%% forked from https://gits-15.sys.kth.se/giampi/kthlatex kthlatex-0.2rc4 on 2020-02-13
%% expanded upon by Gerald Q. Maguire Jr.
%% This template has been adapted by Anders Sjögren to the University
%% Engineering Program in Computer Science at KTH ICT. This adaptation was to
%% translation of English headings into Swedish as the addition of Swedish.
%% Many thanks to others who have provided constructive input regarding the template.

%% Make it possible to conditionally depend on the TeX engine used
\RequirePackage{ifxetex}
\RequirePackage{ifluatex}
\newif\ifxeorlua
\ifxetex\xeorluatrue\fi
\ifluatex\xeorluatrue\fi

\ifxeorlua
%% The following is to ensure that the PDF uses a recent version rather than the typical PDF 1-5
%% This same version of PDF should be set as an option for hyperef

\RequirePackage{expl3}
\ExplSyntaxOn
%pdf_version_gset:n{2.0}
%\pdf_version_gset:n{1.5}

%% Alternatively, if you have a LaTeX newer than June 2022, you can use the following. However, then you have to remove the pdfversion from hyperef. It also breaks hyperxmp. So perhaps it is too early to try using it!
%\DocumentMetadata
%{
%% testphase = phase-I, % tagging without paragraph tagging
% testphase = phase-II % tagging with paragraph tagging and other new stuff.
%pdfversion = 2.0 % pdfversion must be set here.
%}

%% Optionally, you can set the uncompress flag to make it easier to examine the PDF
%\pdf_uncompress: % to check the pdf
\ExplSyntaxOff
\else
\RequirePackage{expl3}
\ExplSyntaxOn
%\pdf_version_gset:n{2.0}
\pdf_version_gset:n{1.5}
\ExplSyntaxOff
\fi


%%%%%%%%%%%%%%%%%%%%%%%%%%%%%%%%%%%%%%%%%%%%%%%%%%%%%%%%%%%%%%%%%%%%%%%%%%%%%%%%%%%%%%%%
%%                              DOCUMENT CLASS
%%%%%%%%%%%%%%%%%%%%%%%%%%%%%%%%%%%%%%%%%%%%%%%%%%%%%%%%%%%%%%%%%%%%%%%%%%%%%%%%%%%%%%%%
%% The template is designed to handle a thesis in English or Swedish
%% set the default language to english or swedish by passing an option to the documentclass - this handles the inside tile page
%% To optimize for digital output (this changes the color palette add the option: digitaloutput
%% To use \ifnomenclature add the option nomenclature
%% To use bibtex or biblatex - include one of these as an option
\documentclass[nomenclature, english, bibtex]{kththesis}
%\documentclass[swedish, biblatex]{kththesis}
% if pdflatex \usepackage[utf8]{inputenc}

%%%%%%%%%%%%%%%%%%%%%%%%%%%%%%%%%%%%%%%%%%%%%%%%%%%%%%%%%%%%%%%%%%%%%%%%%%%%%%%%%%%%%%%%
%%                                 TODO NOTES
%%%%%%%%%%%%%%%%%%%%%%%%%%%%%%%%%%%%%%%%%%%%%%%%%%%%%%%%%%%%%%%%%%%%%%%%%%%%%%%%%%%%%%%%
%% Conventions for todo notes:
%% Informational
%% \generalExpl{Comments/directions/... in English}
\newcommand*{\generalExpl}[1]{\todo[inline]{#1}}                

%% Language-specific information (currently in English or Swedish)
\newcommand*{\engExpl}[1]{\todo[inline, backgroundcolor=kth-lightgreen40]{#1}} %% \engExpl{English descriptions about formatting}
\newcommand*{\sweExpl}[1]{\todo[inline, backgroundcolor=kth-lightblue40]{#1}}  %% % \sweExpl{Text på svenska}

%% warnings
\newcommand*{\warningExpl}[1]{\todo[inline, backgroundcolor=kth-lightred40]{#1}} %% \warningExpl{warnings}

%% Uncomment to hide specific comments, to hide **all** ToDos add `final` to
%% document class
% \renewcommand\warningExpl[1]{}
% \renewcommand\generalExpl[1]{}
% \renewcommand\engExpl[1]{}
%% For example uncommenting the following line hides the Swedish language explanations
% \renewcommand\sweExpl[1]{}

%%%%%%%%%%%%%%%%%%%%%%%%%%%%%%%%%%%%%%%%%%%%%%%%%%%%%%%%%%%%%%%%%%%%%%%%%%%%%%%%%%%%%%%%
%%                               BIBLIOGRAPHY STYLE
%%%%%%%%%%%%%%%%%%%%%%%%%%%%%%%%%%%%%%%%%%%%%%%%%%%%%%%%%%%%%%%%%%%%%%%%%%%%%%%%%%%%%%%%
% \usepackage[style=numeric,sorting=none,backend=biber]{biblatex}
\ifbiblatex
    %\usepackage[language=english,bibstyle=authoryear,citestyle=authoryear, maxbibnames=99]{biblatex}
    %% alternatively you might use another style, such as IEEE and use citestyle=numeric-comp  to put multiple citations in a single pair of square brackets
    \usepackage[style=ieee,citestyle=numeric-comp]{biblatex}
    \addbibresource{references.bib}
    %\DeclareLanguageMapping{norsk}{norwegian}
\else
    %% The line(s) below are for BibTeX
    \bibliographystyle{bibstyle/myIEEEtran}
    %\bibliographystyle{apalike}
\fi

%%%%%%%%%%%%%%%%%%%%%%%%%%%%%%%%%%%%%%%%%%%%%%%%%%%%%%%%%%%%%%%%%%%%%%%%%%%%%%%%%%%%%%%%
%%                             INCLUDES /LIB PACKAGES
%%%%%%%%%%%%%%%%%%%%%%%%%%%%%%%%%%%%%%%%%%%%%%%%%%%%%%%%%%%%%%%%%%%%%%%%%%%%%%%%%%%%%%%%
%% include a variety of packages that are useful
\input{lib/includes}
\input{lib/kthcolors}

%\glsdisablehyper
%\makeglossaries
%\makenoidxglossaries
%\input{lib/acronyms}                %%load the acronyms file

\input{lib/defines}  %% load some additional definitions to make writing more consistent

%%%%%%%%%%%%%%%%%%%%%%%%%%%%%%%%%%%%%%%%%%%%%%%%%%%%%%%%%%%%%%%%%%%%%%%%%%%%%%%%%%%%%%%%
%%                                DIVA COMMANDS
%%%%%%%%%%%%%%%%%%%%%%%%%%%%%%%%%%%%%%%%%%%%%%%%%%%%%%%%%%%%%%%%%%%%%%%%%%%%%%%%%%%%%%%%
%% The following is needed in conjunction with generating the DiVA data with abstracts and keywords using the scontents package and a modified listings environment
%\usepackage{listings}   %%  already included
\ExplSyntaxOn
\newcommand\typestoredx[2]{\expandafter\__scontents_typestored_internal:nn\expandafter{#1} {#2}}
\ExplSyntaxOff
\makeatletter
\let\verbatimsc\@undefined
\let\endverbatimsc\@undefined
\lst@AddToHook{Init}{\hyphenpenalty=50\relax}
\makeatother


\lstnewenvironment{verbatimsc}
    {
    \lstset{%
        basicstyle=\ttfamily\tiny,
        backgroundcolor=\color{white},
        %basicstyle=\tiny,
        %columns=fullflexible,
        columns=[l]fixed,
        language=[LaTeX]TeX,
        %numbers=left,
        %numberstyle=\tiny\color{gray},
        keywordstyle=\color{red},
        breaklines=true,                 %% sets automatic line breaking
        breakatwhitespace=true,          %% sets if automatic breaks should only happen at whitespace
        %keepspaces=false,
        breakindent=0em,
        %fancyvrb=true,
        frame=none,                     %% turn off any box
        postbreak={}                    %% turn off any hook arrow for continuation lines
    }
}{}

%% Add some more keywords to bring out the structure more
\lstdefinestyle{[LaTeX]TeX}{
morekeywords={begin, todo, textbf, textit, texttt}
}

%% definition of new command for bytefield package
\newcommand{\colorbitbox}[3]{%
	\rlap{\bitbox{#2}{\color{#1}\rule{\width}{\height}}}%
	\bitbox{#2}{#3}}


%%%%%%%%%%%%%%%%%%%%%%%%%%%%%%%%%%%%%%%%%%%%%%%%%%%%%%%%%%%%%%%%%%%%%%%%%%%%%%%%%%%%%%%%
%%                             LEFT ALIGNED TABLE CELL
%%%%%%%%%%%%%%%%%%%%%%%%%%%%%%%%%%%%%%%%%%%%%%%%%%%%%%%%%%%%%%%%%%%%%%%%%%%%%%%%%%%%%%%%
%% define a left aligned table cell that is ragged right
\newcolumntype{L}[1]{>{\raggedright\let\newline\\\arraybackslash\hspace{0pt}}p{#1}}

%%%%%%%%%%%%%%%%%%%%%%%%%%%%%%%%%%%%%%%%%%%%%%%%%%%%%%%%%%%%%%%%%%%%%%%%%%%%%%%%%%%%%%%%
%%                         BACKREF BIBLATEX INCOMPATIBILITY
%%%%%%%%%%%%%%%%%%%%%%%%%%%%%%%%%%%%%%%%%%%%%%%%%%%%%%%%%%%%%%%%%%%%%%%%%%%%%%%%%%%%%%%%
%% Because backref is not compatible with biblatex
\ifbiblatex
    \usepackage[plainpages=false]{hyperref}
\else
    \usepackage[
    backref=page,
    pagebackref=false,
    plainpages=false,
                            %% PDF related options
    unicode=true,           %% Unicode encoded PDF strings
    bookmarks=true,         %% generate bookmarks in PDF files
    bookmarksopen=false,    %% Do not automatically open the bookmarks in the PDF reading program
    pdfpagemode=UseNone,    %% None, UseOutlines, UseThumbs, or FullScreen
    destlabel,              %% better naming of destinations
    pdfencoding=auto,       %% for unicode in 
    ]{hyperref}
    \makeatletter
    \ltx@ifpackageloaded{attachfile2}{
    %% cannot use backref if one is using attachfile
    }
    {\usepackage{backref}
    %%
    %% Customize list of backreferences.
    %% From https://tex.stackexchange.com/a/183735/1340
    \renewcommand*{\backref}[1]{}
    \renewcommand*{\backrefalt}[4]{%
    \ifcase #1%
          \or [Page~#2.]%
          \else [Pages~#2.]%
    \fi%
    }
    }
    \makeatother

\fi
\usepackage[all]{hypcap}	%% prevents an issue related to hyperref and caption linking

%%%%%%%%%%%%%%%%%%%%%%%%%%%%%%%%%%%%%%%%%%%%%%%%%%%%%%%%%%%%%%%%%%%%%%%%%%%%%%%%%%%%%%%%
%%                                 ACRONYMS
%%%%%%%%%%%%%%%%%%%%%%%%%%%%%%%%%%%%%%%%%%%%%%%%%%%%%%%%%%%%%%%%%%%%%%%%%%%%%%%%%%%%%%%%
%% note that nonumberlist - removes the cross references to the pages where the acronym appears
%% note that super will set the descriptions text aligned
%% note that nomain - does not produce a main glossary, thus only acronyms will be in the glossary
%% note that nopostdot - will prevent there being a period at the end of each entry
\usepackage[acronym, style=super, section=section, nonumberlist, nomain,
nopostdot]{glossaries}
\setlength{\glsdescwidth}{0.75\textwidth}
\usepackage[automake]{glossaries-extra}
\ifinswedish
    %\usepackage{glossaries-swedish}
\fi

\input{lib/includes-after-hyperref}

%%%%%%%%%%%%%%%%%%%%%%%%%%%%%%%%%%%%%%%%%%%%%%%%%%%%%%%%%%%%%%%%%%%%%%%%%%%%%%%%%%%%%%%%
%%                          MAKE GLOSSARIES COMMAND
%%%%%%%%%%%%%%%%%%%%%%%%%%%%%%%%%%%%%%%%%%%%%%%%%%%%%%%%%%%%%%%%%%%%%%%%%%%%%%%%%%%%%%%%
%\glsdisablehyper
\makeglossaries
%\makenoidxglossaries

%%%%%%%%%%%%%%%%%%%%%%%%%%%%%%%%%%%%%%%%%%%%%%%%%%%%%%%%%%%%%%%%%%%%%%%%%%%%%%%%%%%%%%%%
%%                          PDFLaTeX non-breaking hypen problem
%%%%%%%%%%%%%%%%%%%%%%%%%%%%%%%%%%%%%%%%%%%%%%%%%%%%%%%%%%%%%%%%%%%%%%%%%%%%%%%%%%%%%%%%
%% The following bit of ugliness is because of the problems PDFLaTeX has handling a non-breaking hyphen
%% unless it is converted to UTF-8 encoding.
%% If you do not use such characters in your acronyms, this could be simplified to just include the acronyms file.
\ifxeorlua
\input{lib/acronyms}                %load the acronyms file
\else
%%% Local Variables:
%%% mode: latex
%%% TeX-master: t
%%% End:
%% The following command is used with glossaries-extra
\setabbreviationstyle[acronym]{long-short}
%% The form of the entries in this file is \newacronym{label}{acronym}{phrase}
%%                                      or \newacronym[options]{label}{acronym}{phrase}
%% see "User Manual for glossaries.sty" for the  details about the options, one example is shown below
%% note the specification of the long form plural in the line below
%%\newacronym[longplural={Debugging Information Entities}]{DIE}{DIE}{Debugging Information Entity}
%%
%% The following example also uses options
%%\newacronym[shortplural={OSes}, firstplural={operating systems (OSes)}]{OS}{OS}{operating system}
%%
%% note the use of a non-breaking dash in long text for the following acronym
%%\newacronym{IQL}{IQL}{Independent Q^^e2^^80^^91Learning}
%%
%% Notes
%% 1. you can't use the \gls() command in a heading - but you can get the short (\glsentryshort) 
%% or long version (\glsentryshort) or \glsentrylong or even the text entry (\glsentrytext) and then there is no problem

\newacronym{KTH}{KTH}{KTH Royal Institute of Technology}
\newacronym{ACID}{ACID}{Atomicity, Consistency, Isolation and Durability}
\newacronym{AI}{AI}{Artificial Intelligence}
\newacronym{ML}{ML}{Machine Learning}
\newacronym{BI}{BI}{Business Intelligence}
\newacronym[shortplural={RDDs}, firstplural={Resilient Distributed Datasets (RDDs)}]{RDD}{RDD}{Resilient Distributed Dataset}
\newacronym{OLAP}{OLAP}{On-Line Analytical Processing}
\newacronym{ELT}{ELT}{Extract Load Transform}
\newacronym{ETL}{ETL}{Extract Transform Load}
\newacronym{HDFS}{HDFS}{Hadoop Distributed File System}
\newacronym{JVM}{JVM}{Java Virtual Machine}
\newacronym[shortplural={INs}, firstplural={Industrial Needs (INs)}]{IN}{IN}{Industrial Need}
\newacronym[shortplural={PAs}, firstplural={Project Assumptions (PAs)}]{PA}{PA}{Project Assumption}
\newacronym[shortplural={APIs}, firstplural={Application Programming Interfaces (APIs)}]{API}{API}{Application Programming Interface}
\newacronym{OLTP}{OLTP}{On-Line Transaction Processing}
\newacronym{DBMS}{DBMS}{Data Base Management System}
\newacronym[shortplural={Gs}, firstplural={Goals}]{G}{G}{Goal}
\newacronym[shortplural={RQs}, firstplural={Research Questions}]{RQ}{RQ}{Research Question}
\newacronym[shortplural={Ds}, firstplural={Deliverables}]{D}{D}{Deliverable}
\fi

%%%%%%%%%%%%%%%%%%%%%%%%%%%%%%%%%%%%%%%%%%%%%%%%%%%%%%%%%%%%%%%%%%%%%%%%%%%%%%%%%%%%%%%%
%%                                    FRONT PAGE
%%%%%%%%%%%%%%%%%%%%%%%%%%%%%%%%%%%%%%%%%%%%%%%%%%%%%%%%%%%%%%%%%%%%%%%%%%%%%%%%%%%%%%%%
%% insert the configuration information with author(s), examiner, supervisor(s), ...
%%%%%%%%%%%%%%%%%%%%%%%%%%%%%%%%%%%%%%%%%%%%%%%%%%%%%%%%%%%%%%%%%%%%%%%%%%%%%%%%%%%%%%%%
%%                        INFORMATION INSIDE TITLE PAGE
%%%%%%%%%%%%%%%%%%%%%%%%%%%%%%%%%%%%%%%%%%%%%%%%%%%%%%%%%%%%%%%%%%%%%%%%%%%%%%%%%%%%%%%%
%%
%% AUTHORS INFO
%%
\authorsLastname{Manfredi}
\authorsFirstname{Giovanni}
\email{gioman@kth.se}
\kthid{u142pmki}
%% As per email from KTH Biblioteket on 2021-06-28 students cannot have an OrCiD reported for their degree project
\authorsSchool{\schoolAcronym{EECS}}
%% If the student is not in Stockholm, Sweden, add that information here
%% This information will be used when generating the acknowledgements signature.
%\authorCity{A City}
%\authorCountry{A Country}
%% pass into \authorCityCountryDate{} the month and year for the acknowledgment
%% Specify the month and year for the first author:
\authorCityCountryDate{June 2024}

%%
%% SUPERVISOR INFO
%%
\supervisorAsLastname{Sheikholeslami}
\supervisorAsFirstname{Sina}
\supervisorAsEmail{sinash@kth.se}
%% If the supervisor is from within KTH add their KTHID, School and Department info
\supervisorAsKTHID{u1znylhw}
\supervisorAsSchool{\schoolAcronym{EECS}}
\supervisorAsDepartment{Computer Science}
%% other for a supervisor outside of KTH add their organization info
%\supervisorAsOrganization{Timbuktu University, Department of Pseudoscience}

%%If there is a second supervisor add them here:
\supervisorBsLastname{Schmidt}
\supervisorBsFirstname{Fabian}
\supervisorBsEmail{fschm@kth.se}
%% If the supervisor is from within KTH add their KTHID, School and Department info
\supervisorBsKTHID{u1mrsz0u}
\supervisorBsSchool{\schoolAcronym{EECS}}
\supervisorBsDepartment{Computer Science}
%% other for a supervisor outside of KTH add their organization info
%\supervisorBsOrganization{Timbuktu University, Department of Pseudoscience}

%%
%% EXAMINER INFO
%%
\examinersLastname{Vlassov}
\examinersFirstname{Vladimir}
\examinersEmail{vladv@kth.se}
%% If the examiner is from within KTH add their KTHID, School and Department info
\examinersKTHID{u19yb2c8}
\examinersSchool{\schoolAcronym{EECS}}
\examinersDepartment{Computer Science}
%% other for a examiner outside of KTH add their organization info
%\examinersOrganization{Timbuktu University, Department of Pseudoscience}


\hostcompany{Hopsworks AB} % Remove this line if the project was not done at a host company
%\hostorganization{CERN}   % if there was a host organization

\date{\today}

%%
%% COURSE INFO
%%
\courseCycle{2}
\courseCode{DA258X}
\courseCredits{30.0}

\programcode{TIVNM}
\degreeName{Master's Programme, Distributed Systems and Data Mining for Big Data, 120 credits}
%% ALTERNATIVE
%\degreeName{Master's Programm, ICT Innovation, 120 credits}
\subjectArea{?}
%\subjectArea{Data Science}
%% if there is a second degree
%\secondProgramcode{CINTE}
%\secondDegreeName{test second degree}
%\secondSubjectArea{test second subject area}

%% Note that in the case of both Both Degree of Master of Science in Engineering and Master's degree
%% there are two cases: "Both" is used when the field of technology (\subjectArea{}) and the main subject (\secondSubjectArea{} are different and the case "Same" when they are the same.
%% Both case
%\courseCycle{2}
%\courseCode{xxxxxx}
%\courseCredits{30.0}
%\degreeName{Both Degree of Master of Science in Engineering and Master's degree}
%\subjectArea{Biotechnology}
%\secondSubjectArea{Medical Engineering}

%\courseCycle{2}
%\courseCode{xxxxxx}
%\courseCredits{30.0}
%\degreeName{Både civilingenjörsexamen och masterexamen}
%\subjectArea{bioteknik}
%\secondSubjectArea{medicinsk teknik}

%% Same case
%\courseCycle{2}
%\courseCode{xxxxxx}
%\courseCredits{30.0}
%\degreeName{Both Degree of Master of Science in Engineering and Master's degree}
%\subjectArea{Biotechnology}
%\secondSubjectArea{Biotechnology}

%\courseCycle{2}
%\courseCode{xxxxxx}
%\courseCredits{30.0}
%\degreeName{Både civilingenjörsexamen och masterexamen}
%\subjectArea{bioteknik}
%\secondSubjectArea{bioteknik}

%% For a CDATE student the following are likely values:
%\programcode{CDATE}
%\courseCycle{2}
%\courseCode{DA231X}
%\courseCredits{30.0}
%\degreeName{Degree of Master of Science in Engineering}
%\subjectArea{Computer Science and Engineering}

%% For a TCSCM student the following are likely values:
%\programcode{TCSCM}
%\courseCycle{2}
%\courseCode{DA231X}
%\courseCredits{30.0}
%\degreeName{Master's Programme, Computer Science, 120 credits}
%\subjectArea{Computer Science}

%% For a CMETE student the following are likely values:
%\programcode{CMETE}
%\courseCycle{2}
%\courseCode{DA231X}
%\courseCredits{30.0}
%\degreeName{Degree of Master of Science in Engineering}
%\subjectArea{Media Technology}

%% For a CINTE student the following are likely values:
%\programcode{CINTE}
%\courseCycle{2}
%\courseCode{DA231X}
%\courseCredits{30.0}
%\degreeName{Degree of Master of Science in Engineering}
%\subjectArea{Information and Communication Technology}


%%%%% for DiVA's National Subject Category information
%%% Enter one or more 3 or 5 digit codes
%%% See https://www.scb.se/contentassets/3a12f556522d4bdc887c4838a37c7ec7/standard-for-svensk-indelning--av-forskningsamnen-2011-uppdaterad-aug-2016.pdf
%%% See https://www.scb.se/contentassets/10054f2ef27c437884e8cde0d38b9cc4/oversattningsnyckel-forskningsamnen.pdf
%%%%
%%%% Some examples of these codes are shown below:
% 102 Data- och informationsvetenskap (Datateknik)    Computer and Information Sciences
% 10201 Datavetenskap (datalogi) Computer Sciences 
% 10202 Systemvetenskap, informationssystem och informatik (samhällsvetenskaplig inriktning under 50804)
% Information Systems (Social aspects to be 50804)
% 10203 Bioinformatik (beräkningsbiologi) (tillämpningar under 10610)
% Bioinformatics (Computational Biology) (applications to be 10610)
% 10204 Människa-datorinteraktion (interaktionsdesign) (Samhällsvetenskapliga aspekter under 50803) Human Computer Interaction (Social aspects to be 50803)
% 10205 Programvaruteknik Software Engineering
% 10206 Datorteknik Computer Engineering
% 10207 Datorseende och robotik (autonoma system) Computer Vision and Robotics (Autonomous Systems)
% 10208 Språkteknologi (språkvetenskaplig databehandling) Language Technology (Computational Linguistics)
% 10209 Medieteknik Media and Communication Technology
% 10299 Annan data- och informationsvetenskap Other Computer and Information Science
%%%
% 202 Elektroteknik och elektronik Electrical Engineering, Electronic Engineering, Information Engineering
% 20201 Robotteknik och automation Robotics
% 20202 Reglerteknik Control Engineering
% 20203 Kommunikationssystem Communication Systems
% 20204 Telekommunikation Telecommunications
% 20205 Signalbehandling Signal Processing
% 20206 Datorsystem Computer Systems
% 20207 Inbäddad systemteknik Embedded Systems
% 20299 Annan elektroteknik och elektronik Other Electrical Engineering, Electronic Engineering, Information Engineering
%% Example for a thesis in Computer Science and Computer Systems
\nationalsubjectcategories{10201, 10206}

%%%%%%%%%%%%%%%%%%%%%%%%%%%%%%%%%%%%%%%%%%%%%%%%%%%%%%%%%%%%%%%%%%%%%%%%%%%%%%%%%%%%%%%%
%%                        TITLE OF TITLE PAGE
%%%%%%%%%%%%%%%%%%%%%%%%%%%%%%%%%%%%%%%%%%%%%%%%%%%%%%%%%%%%%%%%%%%%%%%%%%%%%%%%%%%%%%%%
\title{Faster Delta Lake operations using Rust}
\subtitle{How Delta-rs beats Spark in a small scale Feature Store}

%% give the alternative title - i.e., if the thesis is in English, then give a Swedish title
\alttitle{Detta är den svenska översättningen av titeln}
\altsubtitle{Detta är den svenska översättningen av undertiteln}
%% alternative, if the thesis is in Swedish, then give an English title
%\alttitle{This is the English translation of the title}
%\altsubtitle{This is the English translation of the subtitle}

%% Enter the English and Swedish keywords here for use in the PDF meta data _and_ for later use
%% following the respective abstract.
%% Try to put the words in the same order in both languages to facilitate matching. For example:
\EnglishKeywords{Canvas Learning Management System, Docker containers, Performance tuning}
\SwedishKeywords{Canvas Lärplattform, Dockerbehållare, Prestandajustering}

%%%%% For the oral presentation
%% Add this information once your examiner has scheduled your oral presentation
\presentationDateAndTimeISO{2022-03-15 13:00}
\presentationLanguage{eng}
\presentationRoom{via Zoom https://kth-se.zoom.us/j/ddddddddddd}
\presentationAddress{Isafjordsgatan 22 (Kistagången 16)}
\presentationCity{Stockholm}

%% When there are multiple opponents, separate their names with '\&'
%% Opponent's information
\opponentsNames{A. B. Normal \& A. X. E. Capion}

%% Once a thesis is approved by the examiner, add the TRITA number
%% The TRITA number for a thesis consists of two parts a series (unique to each school)
%% and the number in the series which is formatted as the year followed by a colon and
%% then a unique series number for the thesis - starting with 1 each year.
\trita{TRITA -- EECS-EX}{2023:0000}

%% Put the title, author, and keyword information into the PDF meta information
\input{lib/pdf_related_includes}

%%%%%%%%%%%%%%%%%%%%%%%%%%%%%%%%%%%%%%%%%%%%%%%%%%%%%%%%%%%%%%%%%%%%%%%%%%%%%%%%%%%%%%%%
%%                                    CUSTOM COLOURS
%%%%%%%%%%%%%%%%%%%%%%%%%%%%%%%%%%%%%%%%%%%%%%%%%%%%%%%%%%%%%%%%%%%%%%%%%%%%%%%%%%%%%%%%
%% the custom colors and the commands are defined in defines.tex    
\hypersetup{
	colorlinks  = true,
	breaklinks  = true,
	linkcolor   = \linkscolor,
	urlcolor    = \urlscolor,
	citecolor   = \refscolor,
	anchorcolor = black
}

%%%%%%%%%%%%%%%%%%%%%%%%%%%%%%%%%%%%%%%%%%%%%%%%%%%%%%%%%%%%%%%%%%%%%%%%%%%%%%%%%%%%%%%%
%%                             NOMENCLATURE (SYMBOL USED)
%%%%%%%%%%%%%%%%%%%%%%%%%%%%%%%%%%%%%%%%%%%%%%%%%%%%%%%%%%%%%%%%%%%%%%%%%%%%%%%%%%%%%%%%
\ifnomenclature
%% The following lines make the page numbers and equations hyperlinks in the Nomenclature list
\renewcommand*{\pagedeclaration}[1]{\unskip, \dotfill\hyperlink{page.#1}{page\nobreakspace#1}}
%% The following does not work correctly, as the name of the cross-reference is incorrect
%\renewcommand*{\eqdeclaration}[1]{, see equation\nobreakspace(\hyperlink{equation.#1}{#1})}

%% You can also change the page heading for the nomenclature
\renewcommand{\nomname}{List of Symbols Used}

%% You can even add customization text before the list
\renewcommand{\nompreamble}{The following symbols will be later used within the body of the thesis.}
\makenomenclature
\fi

%%%%%%%%%%%%%%%%%%%%%%%%%%%%%%%%%%%%%%%%%%%%%%%%%%%%%%%%%%%%%%%%%%%%%%%%%%%%%%%%%%%%%%%%
%%                               JSON, XML LISTINGS
%%%%%%%%%%%%%%%%%%%%%%%%%%%%%%%%%%%%%%%%%%%%%%%%%%%%%%%%%%%%%%%%%%%%%%%%%%%%%%%%%%%%%%%%
%% format for JSON listings
\colorlet{punct}{red!60!black}
\definecolor{delim}{RGB}{20,105,176}
\definecolor{numb}{RGB}{106, 109, 32}
\definecolor{string}{RGB}{0, 0, 0}

\lstdefinelanguage{json}{
    numbers=none,
    numberstyle=\small,
    frame=none,
    rulecolor=\color{black},
    showspaces=false,
    showtabs=false,
    breaklines=true,
    postbreak=\raisebox{0ex}[0ex][0ex]{\ensuremath{\color{gray}\hookrightarrow\space}},
    breakatwhitespace=true,
    basicstyle=\ttfamily\small,
    extendedchars=false,
    upquote=true,
    morestring=[b]",
    stringstyle=\color{string},
    literate=
     *{0}{{{\color{numb}0}}}{1}
      {1}{{{\color{numb}1}}}{1}
      {2}{{{\color{numb}2}}}{1}
      {3}{{{\color{numb}3}}}{1}
      {4}{{{\color{numb}4}}}{1}
      {5}{{{\color{numb}5}}}{1}
      {6}{{{\color{numb}6}}}{1}
      {7}{{{\color{numb}7}}}{1}
      {8}{{{\color{numb}8}}}{1}
      {9}{{{\color{numb}9}}}{1}
      {:}{{{\color{punct}{:}}}}{1}
      {,}{{{\color{punct}{,}}}}{1}
      {\{}{{{\color{delim}{\{}}}}{1}
      {\}}{{{\color{delim}{\}}}}}{1}
      {[}{{{\color{delim}{[}}}}{1}
      {]}{{{\color{delim}{]}}}}{1}
      {’}{{\char13}}1,
}

\lstdefinelanguage{XML}
{
  basicstyle=\ttfamily\color{blue}\bfseries\small,
  morestring=[b]",
  morestring=[s]{>}{<},
  morecomment=[s]{<?}{?>},
  stringstyle=\color{black},
  identifierstyle=\color{blue},
  keywordstyle=\color{cyan},
  breaklines=true,
  postbreak=\raisebox{0ex}[0ex][0ex]{\ensuremath{\color{gray}\hookrightarrow\space}},
  breakatwhitespace=true,
  morekeywords={xmlns,version,type}% list your attributes here
}

%% In case you use both listings and lstlistings - this makes them both use the same counter
\makeatletter
\AtBeginDocument{\let\c@listing\c@lstlisting}
\makeatother
\usepackage{subfiles}

%%%%%%%%%%%%%%%%%%%%%%%%%%%%%%%%%%%%%%%%%%%%%%%%%%%%%%%%%%%%%%%%%%%%%%%%%%%%%%%%%%%%%%%%
%%                            CREATIVE COMMONS LICENCE
%%%%%%%%%%%%%%%%%%%%%%%%%%%%%%%%%%%%%%%%%%%%%%%%%%%%%%%%%%%%%%%%%%%%%%%%%%%%%%%%%%%%%%%%
%% To have Creative Commons (CC) license and logos use the doclicense package
%% Note that the lowercase version of the license has to be used in the modifier
%% i.e., one of by, by-nc, by-nd, by-nc-nd, by-sa, by-nc-sa, zero.
%% For background see:
%% https://www.kb.se/samverkan-och-utveckling/oppen-tillgang-och-bibsamkonsortiet/open-access-and-bibsam-consortium/open-access/creative-commons-faq-for-researchers.html
%% https://kib.ki.se/en/publish-analyse/publish-your-article-open-access/open-licence-your-publication-cc
\begin{comment}
\usepackage[
    type={CC},
    %modifier={by-nc-nd},
    %version={4.0},
    modifier={by-nc},
    imagemodifier={-eu-88x31},  % to get Euro symbol rather than Dollar sign
    hyphenation={RaggedRight},
    version={4.0},
    %modifier={zero},
    %version={1.0},
]{doclicense}
\end{comment}

%%%%%%%%%%%%%%%%%%%%%%%%%%%%%%%%%%%%%%%%%%%%%%%%%%%%%%%%%%%%%%%%%%%%%%%%%%%%%%%%%%%%%%%%
%%                                DOCUMENT
%%%%%%%%%%%%%%%%%%%%%%%%%%%%%%%%%%%%%%%%%%%%%%%%%%%%%%%%%%%%%%%%%%%%%%%%%%%%%%%%%%%%%%%%
\begin{document}
%\selectlanguage{swedish}
\selectlanguage{english}

%% Set the numbering for the title page to a numbering series not in the preface or body
\pagenumbering{alph}
\kthcover
\clearpage\thispagestyle{empty}\mbox{} % empty back of front cover
\titlepage

%% If you do not want to have a bookinfo page, comment out the line saying \bookinfopage and add a \cleardoublepage
%% If you want a bookinfo page: you will get a copyright notice, unless you have used the doclicense package in which case you will get a Creative Commons license. To include the doclicense package, uncomment the configuration of this package above and configure it with your choice of license.
\bookinfopage

%% Frontmatter includes the abstracts and table-of-contents
\frontmatter
\setcounter{page}{1}

%%%%%%%%%%%%%%%%%%%%%%%%%%%%%%%%%%%%%%%%%%%%%%%%%%%%%%%%%%%%%%%%%%%%%%%%%%%%%%%%%%%%%%%%
%%                                ABSTRACTS
%%%%%%%%%%%%%%%%%%%%%%%%%%%%%%%%%%%%%%%%%%%%%%%%%%%%%%%%%%%%%%%%%%%%%%%%%%%%%%%%%%%%%%%%
\begin{abstract}
    % The first abstract should be in the language of the thesis.
  \markboth{\abstractname}{}
\begin{scontents}[store-env=lang]
eng
\end{scontents}
%%% The contents of the abstract (between the begin and end of scontents) will be saved in LaTeX format
%%% and output on the page(s) at the end of the thesis with information for DiVA facilitating the correct
%%% entry of the meta data for your thesis.
%%% These page(s) will be removed before the thesis is inserted into DiVA.
%% \engExpl{All theses at KTH are \textbf{required} to have an abstract in both \textit{English} and \textit{Swedish}.}
%% \engExpl{Exchange students may want to include one or more abstracts in the language(s) used in their home institutions to avoid the need to write another thesis when returning to their home institution.}
%% \generalExpl{Keep in mind that most of your potential readers are only going to read your \texttt{title} and \texttt{abstract}. This is why the abstract must give them enough information so that they can decide if this document is relevant to them or not. Otherwise, the likely default choice is to ignore the rest of your document.\\
%% An abstract should stand on its own, i.e., no citations, cross-references to the body of the document, acronyms must be spelled out, \ldots .\\Write this early and revise as necessary. This will help keep you focused on what you are trying to do.}
\begin{scontents}[store-env=abstracts,print-env=true]
The need to build Machine Learning (ML) models based on ever-increasing amounts of data brought new challenges to data management systems. Feature stores have emerged as a centralized data platform enabling feature reuse while organizing data transformations and ensuring consistency between feature engineering, model training, and inference. Recent publications demonstrate that the Hopsworks feature store outperforms existing cloud-based alternatives in training and online inference query workloads. In its offline feature store, the Hopsworks feature store stores batch or historical data, collecting it into feature groups, i.e., logical tables of figures, organized in Apache Hudi tables, i.e., a data management layer, and stored on HopsFS, Hopsworks HDFS distribution. However, even in this system, the latency to perform a write operation is at least one or more minutes, even for small quantities of data (1 GB or less). A previous improvement in read latency using an Arrow Flight and DuckDB server suggests that this limitation is caused by Spark, which the system uses to write data on Apache Hudi tables. 
A promising approach to avoid using Spark appears to be adopting Delta Lake over Apache Hudi to manage and access data using a Rust library called delta-rs. This thesis investigates the possibility of reducing the read and write latency in the offline feature store by expanding the delta-rs library to support the Hopsworks feature store file system called HopsFS and comparatively evaluating the performance of the legacy and newly implemented system. Two major iterations of storage support in delta-rs for HopsFS were developed to meet the strict production-ready requirements defined before development. The system was then evaluated by performing and measuring read and write operations in four different CPU configurations, increasing the number of CPU cores up to eight. Experiments were performed fifty times to estimate a confidence interval, allowing an accurate comparative evaluation of the systems. Results confirmed the superior performance of the delta-rs library over the Spark system in all write operations with a latency reduction from ten up to forty times. Delta-rs also surpassed the Spark-alternative in read operations with a latency reduction of forty-seven percent, up to forty times. These findings encourage future research investigating Spark-alternative when optimizing performance in small-scale (1 GB - 100 GB) data management systems. The system developed will find application in the Hopsworks feature store production environment.
\end{scontents}
\begin{comment}
\engExpl{The following are some notes about what can be included (in terms of LaTeX) in your abstract.}
Choice of typeface with \textbackslash textit, \textbackslash textbf, and \textbackslash texttt:  \textit{x}, \textbf{x}, and \texttt{x}.

Text superscripts and subscripts with \textbackslash textsubscript and \textbackslash textsuperscript: A\textsubscript{x} and A\textsuperscript{x}.

Some symbols that you might find useful are available, such as: \textbackslash textregistered, \textbackslash texttrademark, and \textbackslash textcopyright. For example, 
the copyright symbol: \textbackslash textcopyright Maguire 2022 results in \textcopyright Maguire 2022. Additionally, here are some examples of text superscripts (which can be combined with some symbols): \textbackslash textsuperscript\{99m\}Tc, A\textbackslash textsuperscript\{*\}, A\textbackslash textsuperscript\{\textbackslash textregistered\}, and A\textbackslash texttrademark resulting in \textsuperscript{99m}Tc, A\textsuperscript{*}, A\textsuperscript{\textregistered}, and A\texttrademark. Two examples of subscripts are: H\textbackslash textsubscript\{2\}O and CO\textbackslash textsubscript\{2\} which produce  H\textsubscript{2}O and CO\textsubscript{2}.

You can use simple environments with begin and end: itemize and enumerate and within these use instances of \textbackslash item.

The following commands can be used: \textbackslash eg, \textbackslash Eg, \textbackslash ie, \textbackslash Ie, \textbackslash etc, and \textbackslash etal: \eg, \Eg, \ie, \Ie, \etc, and \etal.

The following commands for numbering with lowercase Roman numerals: \textbackslash first, \textbackslash Second, \textbackslash third, \textbackslash fourth, \textbackslash fifth, \textbackslash sixth, \textbackslash seventh, and \textbackslash eighth: \first, \Second, \third, \fourth, \fifth, \sixth, \seventh, and \eighth. Note that the second case is set with a capital 'S' to avoid conflicts with the use of second of as a unit in the \texttt{siunitx} package.

Equations using \textbackslash( xxxx \textbackslash) or \textbackslash[ xxxx \textbackslash] can be used in the abstract. For example: \( (C_5O_2H_8)_n \)
or \[ \int_{a}^{b} x^2 \,dx \]
Note that you \textbf{cannot} use an equation between dollar signs.

Even LaTeX comments can be handled, for example: \% comment.
Note that one can include percentages, such as: 51\% or \SI{51}{\percent}.
\end{comment}
\subsection*{Keywords}
\begin{scontents}[store-env=keywords,print-env=true]
% If you set the EnglishKeywords earlier, you can retrieve them with:
\InsertKeywords{english}
% If you did not set the EnglishKeywords earlier then simply enter the keywords here:
%First keyword, Second keyword, Third keyword, Fourth keyword
\end{scontents}
%%\engExpl{\textbf{Choosing good keywords can help others to locate your paper, thesis, dissertation, \ldots and related work.}}
%%Choose the most specific keyword from those used in your domain, see for example: the ACM Computing Classification System ({\small \url{https://www.acm.org/publications/computing-classification-system/how-to-use})},
%%the IEEE Taxonomy ({\small \url{https://www.ieee.org/publications/services/thesaurus-thank-you.html}}), PhySH (Physics Subject Headings)\linebreak[4] ({\small \url{https://physh.aps.org/}}), \ldots or keyword selection tools such as the  National Library of Medicine's Medical Subject Headings (MeSH)  ({\small \url{https://www.nlm.nih.gov/mesh/authors.html}}) or Google's Keyword Tool ({\small \url{https://keywordtool.io/}})\\

%%\textbf{Formatting the keywords}:
%%\begin{itemize}
%%  \item The first letter of a keyword should be set with a capital letter, and proper names should be capitalized as usual.
%%  \item Spell out acronyms and abbreviations.
%%  \item Avoid "stop words" - as they generally carry little or no information.
%%  \item List your keywords separated by commas (",").
%%\end{itemize}    
%%Since you should have both English and Swedish keywords - you might think of ordering them in corresponding order (\ie, so that the n\textsuperscript{th} word in each list correspond) - this makes it easier to mechanically find matching keywords.
\end{abstract}

\cleardoublepage

\babelpolyLangStart{swedish}
\begin{abstract}
   \markboth{\abstractname}{}
\begin{scontents}[store-env=lang]
swe
\end{scontents}
%%\warningExpl{Inside the following scontents environment, you cannot use a \textbackslash include{filename} as it will not end up in the for diva information. Additionally, you should not use a straight double quote character in the abstracts or keywords, use two single quote characters instead.}
\begin{scontents}[store-env=abstracts,print-env=true]
%%\generalExpl{Enter your Swedish abstract or summary here!}
%%\sweExpl{Alla avhandlingar vid KTH \textbf{måste ha} ett abstrakt på både \textit{engelska} och \textit{svenska}.\\
%%Om du skriver din avhandling på svenska ska detta göras först (och placera det som det första abstraktet) - och du bör revidera det vid behov.}
Behovet av att bygga modeller för maskininlärning (ML) baserat på ständigt ökande datamängder har medfört nya utmaningar för datahanteringssystemen. Feature Stores har uppmärksammats som centraliserade dataplattformar som möjliggör återanvändning av features vid organisering av datatransformationer och säkerställer att feature-utveckling, modellträning och modellinferens är konsistenta. Nya publikationer visar att Hopsworks Feature Store har bättre prestanda än befintliga molnbaserade alternativ under modellträning och under beräkningslast för online-inferens. I sitt offline feature store lagrar Hopsworks feature store batchdata, dvs. historisk data, samlar det i feature groups, dvs. logiska tabeller med features, organiserade i Apache Hudi-tabeller, dvs. ett datahanteringslager, och lagrade i HopsFS, Hopsworks HDFS-distribution.Men även i detta system är latensen för att utföra en skrivoperation minst en eller flera minuter, även för små datamängder (1 GB eller mindre). En tidigare förbättring av läslatens med hjälp av en Arrow Flight- och DuckDB-server tyder på att denna begränsning orsakas av Spark, som systemet använder för att skriva data i Apache Hudi-tabeller. En lovande metod för att undvika Spark verkar vara att använda Delta Lake istället för Apache Hudi för datahantering, och använda ett Rust-bibliotek som heter delta-rs för dataåtkomst. Den här avhandlingen undersöker möjligheten att minska läs- och skrivfördröjningen i offline-funktionslagret genom att utöka delta-rs-biblioteket för att stödja Hopsworks-funktionslagrets filsystem som kallas HopsFS och jämförande utvärdera prestanda för det äldre och det nyligen implementerade systemet. Två större iterationer av utveckling av stöd för HopsFS-lagring i delta-rs utfördes för att uppfylla de strikta krav på produktionsfärdighet som definierades före utvecklingen. Systemet utvärderades sedan genom att utföra och mäta läs- och skrivoperationer i fyra olika CPU-konfigurationer, där antalet CPU-kärnor ökades upp till åtta. Experimenten utfördes femtio gånger för att uppskatta ett konfidensintervall, vilket möjliggjorde en korrekt jämförande utvärdering av systemen.  Resultaten bekräftade delta-rs-bibliotekets överlägsna prestanda jämfört med Spark-systemet i alla skrivoperationer med en latensminskning från tio upp till fyrtio gånger. Delta-rs överträffade också Spark-alternativet i läsoperationer med en latensminskning på fyrtiosju procent, upp till fyrtio gånger.
Dessa resultat uppmuntrar till framtida forskning där Spark alternativet undersöks för att optimera prestanda i småskaliga (1 GB - 100 GB) datahanteringssystem. Det utvecklade systemet kommer att användas i produktionsmiljön för Hopsworks feature store.
%%\engExpl{If you are writing your thesis in English, you can leave this until the draft version that goes to your opponent for the written opposition. In this way, you can provide the English and Swedish abstract/summary information that can be used in the announcement for your oral presentation.\\If you are writing your thesis in English, then this section can be a summary targeted at a more general reader. However, if you are writing your thesis in Swedish, then the reverse is true – your abstract should be for your target audience, while an English summary can be written targeted at a more general audience.\\This means that the English abstract and Swedish sammnfattning  
%%or Swedish abstract and English summary need not be literal translations of each other.}

%%\warningExpl{Do not use the \textbackslash glspl\{\} command in an abstract that is not in English, as my programs do not know how to generate plurals in other languages. Instead, you will need to spell these terms out or give the proper plural form. In fact, it is a good idea not to use the glossary commands at all in an abstract/summary in a language other than the language used in the \texttt{acronyms.tex file} - since the glossary package does \textbf{not} support use of more than one language.}

%%\engExpl{The abstract in the language used for the thesis should be the first abstract, while the Summary/Sammanfattning in the other language can follow}
\end{scontents}
\subsection*{Nyckelord}
\begin{scontents}[store-env=keywords,print-env=true]
% SwedishKeywords were set earlier, hence we can use alternative 2
\InsertKeywords{swedish}
%Första nyckelordet, Andra nyckelordet, Tredje nyckelordet, Fjärde nyckelordet
\end{scontents}
\end{abstract}
\babelpolyLangStop{swedish}

\cleardoublepage

\babelpolyLangStart{italian}
\begin{abstract}
    \markboth{\abstractname}{}
\begin{scontents}[store-env=lang]
ita
\end{scontents}
\begin{scontents}[store-env=abstracts,print-env=true]
La necessità di costruire modelli di \textit{Machine Learning (ML)} basati su quantità sempre maggiori di dati ha posto nuove sfide ai sistemi di gestione dei dati. I \textit{feature stores} sono emersi come una soluzione efficace per consentire il riutilizzo delle \textit{features}, organizzando al contempo le trasformazioni dei dati e garantendo la coerenza tra il \textit{feature engineering}, il \textit{training} e l'\textit{inference} dei modelli. Recenti pubblicazioni dimostrano che il \textit{feature store} di Hopsworks presenta metriche di prestazione superiori sia per quanto riguarda il \textit{training} dei modelli sia per quanto riguarda le \textit{query} di \textit{online inference}, rispetto alle alternative esistenti basate su \textit{cloud}. In questo sistema, la latenza per eseguire un'operazione di scrittura è di almeno uno o più minuti, anche per piccole quantità di dati (1 GB o meno). Si ritiene che questo limite sia specifico di Spark, che il sistema utilizza per scrivere i dati sull'\textit{offline feature store}. Questa ipotesi è già stata confermata nel caso della latenza in lettura, dove la scelta di un'alternativa a Spark, ovvero un server Arrow Flight e DuckDB, ha migliorato notevolmente le prestazioni. Un approccio promettente sembra essere l'adozione di una nuova soluzione per la gestione dei dati, Delta Lake, e l'accesso ad essa tramite una libreria Rust chiamata delta-rs. Questa tesi studia la possibilità di ridurre la latenza di lettura e scrittura nell' \textit{offline feature store}  espandendo la libreria delta-rs per supportare il \textit{file system} del \textit{feature store} di Hopsworks, chiamato HopsFS, e valutando in modo comparativo le prestazioni del sistema precedente e di quello appena implementato. Dopo la prima fase di implementazione iterativa del sistema basata su requisiti fissati, il sistema è stato valutato eseguendo e misurando le operazioni di lettura e scrittura in quattro diverse configurazioni di CPU, aumentando il numero di core della CPU fino a otto. Gli esperimenti sono stati eseguiti cinquanta volte per stimare un intervallo di confidenza che permettesse un'accurata valutazione comparativa dei sistemi. I risultati hanno confermato la superiorità della libreria delta-rs rispetto al sistema Spark in tutte le operazioni di scrittura, con una riduzione di dieci volte della latenza. Delta-rs ha anche superato il sistema alternativo a Spark usato nelle operazioni di lettura, con una riduzione di dieci volte della latenza in tutti gli esperimenti tranne in quello con la tabella più grande (60 milioni di righe), dove il miglioramento è di un fattore minore. Questi risultati incoraggiano la ricerca futura di alternative a Spark per l'ottimizzazione delle prestazioni nei sistemi di gestione dei dati su piccola scala (1 GB - 100 GB).
\end{scontents}
\subsection*{Parole chiave}
\begin{scontents}[store-env=keywords,print-env=true]
Machine Learning, Feature Store, Limitazione specifica di Spark, Delta Lake, libreria delta-rs, Latenza di lettura/scrittura
\end{scontents}
\end{abstract}
\babelpolyLangStop{italian}

\cleardoublepage
% note that a command is used to avoid Overleaf parsing problems

%%%%%%%%%%%%%%%%%%%%%%%%%%%%%%%%%%%%%%%%%%%%%%%%%%%%%%%%%%%%%%%%%%%%%%%%%%%%%%%%%%%%%%%%
%%                               ACKNOWLEDGEMENTS
%%%%%%%%%%%%%%%%%%%%%%%%%%%%%%%%%%%%%%%%%%%%%%%%%%%%%%%%%%%%%%%%%%%%%%%%%%%%%%%%%%%%%%%%
\section*{Acknowledgments}
    \markboth{Acknowledgments}{}
%%\engExpl{It is nice to acknowledge the people that have helped you. It is
%%  also necessary to acknowledge any special permissions that you have gotten –
%%  for example, getting permission from the copyright owner to reproduce a
%%  figure. In this case, you should acknowledge them and this permission here
%%  and in the figure’s caption. \\
%%  Note: If you do \textbf{not} have the copyright owner’s permission, then you \textbf{cannot} use any copyrighted figures/tables/\ldots . Unless stated otherwise all figures/tables/\ldots are generally copyrighted.
%%}

\begin{quote}
    “ Persistence and resilience only come from having been given the chance to work through difficult problems „

    -- Gever Tulley
\end{quote}
The work that brought me to complete this master thesis made me grow incredibly, laying a foundation of the person I am today. This would have not been possible without other individuals, who I would like to thank in this section.

I would like to thank Jim Dowling for giving me the opportunity to work on this project and Salman Niazi that helped me make it happen. Many thanks to also all the Hopsworks AB employees, who welcomed me from day one.

I would like to thank my KTH examiner Vladimir Vlassov, and my two supervisors Fabian Schmidt and Sina Sheikholeslami for their many corrections to my numerous drafts.

I would like to thank all my friends that in the last years made me feel part of a larger family. Especially my Milan group, to which I will never be tired to come back to: Sebastiano, Virginia, Luca, Alfonso, Giacomo and Andrea. A special mention goes to Sebastiano that in the last year was always by my side, an amazing partner in an ever changing world. Even if now we parted ways, I would still love to visit you in any city you might be located in.

I would like to thank my parents and my brothers. In my hardest moment of difficulty they were able to help me get back on track to complete this project, while supporting me all the way. You enabled me to perform my best now and to do so in the future.

Finally I would like to thank my loved one, Elena who has been by my side well before I started my academic path. You were my space of serenity even in the darkest hour. You brought me joy when I felt I couldn't feel any. I look forward to our present and our future together.


\acknowlegmentssignature

%%%%%%%%%%%%%%%%%%%%%%%%%%%%%%%%%%%%%%%%%%%%%%%%%%%%%%%%%%%%%%%%%%%%%%%%%%%%%%%%%%%%%%%%
%%              TABLE OF CONTENTS, LIST OF FIGURES, TABLES, LISTINGS
%%%%%%%%%%%%%%%%%%%%%%%%%%%%%%%%%%%%%%%%%%%%%%%%%%%%%%%%%%%%%%%%%%%%%%%%%%%%%%%%%%%%%%%%
\fancypagestyle{plain}{}
\renewcommand{\chaptermark}[1]{ \markboth{#1}{}} 
%% TABLE OF CONTENTS
\tableofcontents
  \markboth{\contentsname}{}

%\cleardoublepage
%% LIST OF FIGURES
%\listoffigures

%\cleardoublepage
%% LIST OF TABLES
%\listoftables

%\cleardoublepage
%% LIST OF LISTINGS
%\lstlistoflistings

\cleardoublepage
%%%%%%%%%%%%%%%%%%%%%%%%%%%%%%%%%%%%%%%%%%%%%%%%%%%%%%%%%%%%%%%%%%%%%%%%%%%%%%%%%%%%%%%%
%%                            GLOSSARIES, ACRONIMS
%%%%%%%%%%%%%%%%%%%%%%%%%%%%%%%%%%%%%%%%%%%%%%%%%%%%%%%%%%%%%%%%%%%%%%%%%%%%%%%%%%%%%%%%
% Align the text expansion of the glossary entries
\newglossarystyle{mylong}{%
  \setglossarystyle{long}%
  \renewenvironment{theglossary}%
     {\begin{longtable}[l]{@{}p{\dimexpr 2cm-\tabcolsep}p{0.8\hsize}}}% <-- change the value here
     {\end{longtable}}%
 }
%\glsaddall
%\printglossaries[type=\acronymtype, title={List of acronyms}]
\printglossary[style=mylong, type=\acronymtype, title={List of acronyms and abbreviations}]
%\printglossary[type=\acronymtype, title={List of acronyms and abbreviations}]

%\printnoidxglossary[style=mylong, title={List of acronyms and abbreviations}]
%\engExpl{The list of acronyms and abbreviations should be in alphabetical order based on the spelling of the acronym or abbreviation.}

% if the nomenclature option was specified, then include the nomenclature page(s)
\ifnomenclature
    \cleardoublepage
    % Output the nomenclature list
    \printnomenclature
\fi

%% The following label is essential to know the page number of the last page of the preface
%% It is used to compute the data for the "For DIVA" pages
\label{pg:lastPageofPreface}
%%%%%%%%%%%%%%%%%%%%%%%%%%%%%%%%%%%%%%%%%%%%%%%%%%%%%%%%%%%%%%%%%%%%%%%%%%%%%%%%%%%%%%%%
%%                            THESIS CONTENTS
%%%%%%%%%%%%%%%%%%%%%%%%%%%%%%%%%%%%%%%%%%%%%%%%%%%%%%%%%%%%%%%%%%%%%%%%%%%%%%%%%%%%%%%%
%% Mainmatter is where the actual contents of the thesis goes
\mainmatter
\glsresetall
\renewcommand{\chaptermark}[1]{\markboth{#1}{}}
\selectlanguage{english}

\chapter{Introduction}
    \label{ch:introduction}
    Data lakehouse systems are increasingly becoming the primary choice for running analytics in large companies with over 1000 employees \cite{StateDataLakehouse2024}. The data lakehouse architecture \cite{lakehouse2021} is preferred over old paradigms, i.e., data warehouses and data lakes, as it builds upon the advantages of both systems, having the scalability properties of data lakes while preserving the \gls{ACID} properties typical of data warehouses \cite{lakehouse2021}. Additionally, data lakehouse systems include partitioning, which reduces query complexity significantly and provides "time travel" capabilities, enabling users to access different versions of data, versioned over time \cite{crociDataLakehouseHype2022}.

Three main implementations of this paradigm emerged over time \cite{ApacheHudiVs}: 
\begin{enumerate}
    \item \textbf{Apache Hudi}: first introduced by Uber \cite{rajaperumalUberEngineeringIncremental2017}, and now primarily backed by Uber, Tencent, Alibaba, and Bytedance.
    \item \textbf{Apache Iceberg}: first introduced by Netflix, and now primarily backed by Netflix, Apple, and Tencent.
    \item \textbf{Delta Lake}: first introduced by Databricks \cite{armbrustDeltaLakeHighperformance2020}, and now primarily backed by Databricks and Microsoft.
\end{enumerate}

While large communities support all three projects, Delta Lake is acknowledged as the de-facto data lakehouse solution \cite{ApacheHudiVs}. This recognition is mainly thanks to Databricks, which first promoted this new architecture over data lakes among their clients around 2020 \cite{armbrustDeltaLakeHighperformance2020}.

As a data query and processing engine, Delta Lake is typically used with Apache Spark \cite{zahariaApacheSparkUnified2016}. This approach is practical when processing large quantities of data (1 TB or more) in the cloud, but whether this approach is effective on a small scale (1 GB - 100 GB) remains to be investigated \cite{Khazanchi1801362}.

DuckDB~\cite{raasveldtDuckDBEmbeddableAnalytical2019}, a \gls{DBMS} and Polars~\cite{vinkWroteOneFastest2021}, a DataFrame library, highlighted the limitations of Apache Spark. When processing data locally with smaller volumes, an Apache Spark cluster underperforms compared to other alternatives. This result ultimately increases costs and computation time when using Spark~\cite{BenchmarkResultsSpark,ebergenUpdatesH2OAi2023}.

Another important consideration is that Python, due to its simplicity and high level of abstraction, has emerged as the most widely used programming language in the field of data science \cite{nagpalPythonDataAnalytics2019}. Python is currently the most popular general-purpose programming language \cite{TIOBEIndex, StackOverflowDeveloper}, and it is by far the most used language for \gls{ML} and \gls{AI} applications \cite{python-machine-learning}; this is mainly thanks to its strong abstraction capabilities and accessibility. This trend can also be observed by looking at the most popular libraries among developers, where two Python libraries make the podium: NumPy and Pandas \cite{StackOverflowDeveloper}.
In this scenario, using a Python client for Delta Lake would be beneficial as developers would not have to resort to Apache Spark and its Python \gls{API} (PySpark). This approach with small-scale (1 GB - 100 GB) use cases would improve performance significantly.

This native Python access for Delta Lake directly benefits Hopsworks \gls{AB}, the host company of this master thesis. Hopsworks \gls{AB} develops a homonymous feature store for \gls{ML}. This centralized, collaborative data platform enables the storage and access of reusable features~\footnote{Definition from the company's website at \url{https://www.hopsworks.ai/}}. This architecture also supports point-in-time correct datasets from historical feature data \cite{Pettersson1695672}.

This presented project aims to reduce the latency (seconds) and thus increase the data throughput (rows/second) for reading and writing on Delta Lake tables that act as an offline feature store in Hopsworks. Currently, the writing pipeline is Apache Spark-based, and the fundamental hypothesis of the project is that a faster non-Apache Spark alternative is possible. If successful, Hopsworks AB will consider incorporating this system into the open-source Hopsworks feature store, significantly enhancing the experience for Python users working with smaller datasets (1 GB - 100 GB). More generally, this work will outline the possibility of Apache Spark alternatives in small-scale use cases.

This thesis's main contributions are the following:
\begin{itemize}
    \item Two code implementations adding support for \gls{HDFS} and \gls{HopsFS} in the delta-rs library. Of these, the first one is incomplete, as the second was preferred according to the consistency and maintainability requirements defined in Section \ref{subsec:requirements}. These code contributions are more than two thousand \gls{LOC} for the first implementation and eight hundred for the second. Note that while these metrics might provide some insight into the contribution's value, this work's true value is in creating a production-ready solution that correctly navigates a complex data stack of technologies with intricate dependencies, answering all requirements. The most relevant recognition of this contribution is the inclusion of this code implementation in a production environment in the Hopsworks feature store shortly after the thesis publication.
    \item The experiments' results detail the difference in performance between the newly implemented system and the legacy system in read and write operations expressed as latency and throughput. Experiments were also performed at different \gls{CPU} configurations and table sizes fifty times, enabling a confidence interval estimate. Results report that the new system using the delta-rs library to access data has a latency reduction compared to the legacy system from ten up to forty times in write operations and from forty-seven percent up to forty times in read operations. These results are a solid contribution to the data management field, confirming present research on the limitations of using Spark with small amounts of data. Additionally, the value of this work consists of the large number of experiments conducted in a well-defined environment, which enables reproducible results using the code made available in this thesis.
\end{itemize}

\section{Background}
    \label{sec:background}
    Three key aspects of this project are essential to a comprehensive understanding: the development of the data lakehouse architecture, the significance and workflows of Spark, and the emergence of Python as a dominant programming language.

Data lakehouse is a term coined by Databricks in 2020 \cite{WhatLakehouse2020} to define a new design standard emerging in the industry. This new paradigm combined the capability of data lakes in storing and managing unstructured data with the \gls{ACID} properties typical of data warehouses.
Data warehouses became a dominant standard in the '90s, and early 2000s \cite{chaudhuriOverviewDataWarehousing1997}, enabling companies to generate \gls{BI} insights, managing different structured data sources. The problems related to this architecture were highlighted in the 2010s when the need to manage large quantities of unstructured data rose \cite{ederUnstructuredData802008}. 
So data lakes became the pool where all data could be stored, on top of which a more complex architecture could be built, consisting of data warehouses for \gls{BI} and \gls{ML} pipelines.
This architecture, while more suitable for unstructured data, introduces many complexities and costs related to the need to have replicated data (data lake and data warehouse) and several \gls{ELT} and \gls{ETL} computations.
Data lakehouse systems solved the problems of data lakes by implementing data management and performance features on top of open data formats such as Parquet \cite{DremelMadeSimple}. Three key technologies enabled this paradigm: (i) a metadata layer for data lakes, tracking which files are part of different tables; (ii) a new query engine design, providing optimizations such as \gls{RAM}/\gls{SSD} caching; and (iii) an accessible \gls{API} access for \gls{ML} and \gls{AI} applications. Uber first open-sourced this architecture design with Apache Hudi in 2017 \cite{rajaperumalUberEngineeringIncremental2017}, and then Databricks did the same with Delta Lake in 2020 \cite{armbrustDeltaLakeHighperformance2020}.

Spark is a distributed computing framework used to support large-scale data-intensive applications \cite{zaharia2010spark}. Developed as an evolution of the MapReduce paradigm, Spark has become the de-facto standard for big data processing due to its superior performance and versatility. Spark significantly improved its performance compared to its predecessor, i.e., Hadoop MapReduce (10 times better in its first iteration) \cite{zaharia2010spark} thanks to its use of in-memory processing. This feature means that Spark avoids going back and forth between storage disks to store the computation results. Spark, open-sourced under the Apache foundation as Apache Spark (from now on referred to as Spark), has seen widespread success and adoption in various applications, becoming the de-facto data-intensive computing platform for the distributed computing world. While Spark is often used as a comprehensive solution \cite{zahariaApacheSparkUnified2016}, different solutions might be better suited for a specific scenario.
An example of this is the case of Apache Flink \cite{carboneApacheFlinkStream}, designed for real-time data streams, which prevails over Spark where low latency real-time analytics are required. Similarly, Spark might not be the best tool for lower-scale applications where Spark's high-scaling capabilities may not be necessary. This is the case of DuckDB \cite{raasveldtDuckDBEmbeddableAnalytical2019} and Polars \cite{vinkWroteOneFastest2021}, that by focusing on a small-scale data (1 GB - 100 GB) provide a fast \gls{OLAP} embedded database and DataFrame management system respectively offering an overall faster computation compared to starting a Spark cluster for to perform the same operations. These technologies demonstrate that new applications outperforming Spark in specific use cases are possible and already in use. In particular, Apache Flink and DuckDB show that this is possible for real-time data streaming or small-scale computation. In this project, the latter use case is going to be explored.

Python can be considered the primary programming language among data scientists \cite{Python_CS-R9526}. Many first adopted Python thanks to its focus on ease of use, high abstraction level, and readability. These features helped create a fast-growing community behind the project, which led to the development of many libraries and \glspl{API}. So now, more than thirty years after its creation, it has become the de-facto standard for data science thanks to many daily used Python libraries such as TensorFlow, NumPy, SciPy, Pandas, PyTorch, Keras and many others. Python is also considered to be the most popular programming language, according to the number of results by search query (\textit{+"<language> programming"}) in 25 different search engines~\footnote{Evaluation methodology defined at \url{https://www.tiobe.com/tiobe-
index/programminglanguages_definition/}}. This ranking is computed yearly in the TIOBE Index \cite{TIOBEIndex}. The April 2024 rankings reveal that Python holds a rating of 16.41\%, followed by C at 10.21\%. The index also highlights trends from recent years, clearly illustrating Python's rise over traditionally popular languages like C and Java, both of which Python surpassed between 2021 and 2022. These scores underline the importance of providing Python \glspl{API}, particularly for programmers and data scientists, to enhance engagement and expand the capabilities of a framework.

\section{Problem}
    \label{sec:problem}
    The Hopsworks Feature Store \cite{HopsworksBatchRealtime2024} first used Apache Hudi for their Offline Feature Store, as it was the first open-sourced data lakehouse in 2017. Recently, Hopsworks \gls{AB} added support for using Delta Lake as an offline feature store, following its clients' requests. Spark is used as a query engine in the system, i.e., executes the query (read, write, or delete) on the Offline Feature Store. Running the system showed that even a write operation on a small dataset, consisting of 1 GB of data or less, takes one or more minutes to complete.

This hurts Hopsworks' typical use case, which sits between tests on small quantities of data (scale between 1-10 GBs) and production scenarios on a larger scale but still relatively small (scale between 10-100 GBs).

This research's underlying hypothesis is that this slow transaction time is a Spark-specific issue. This has led Hopsworks to adopt Spark alternatives \cite{Khazanchi1801362} for reading in their Apache Hudi system. Delta Lake supports Spark alternatives for accessing and querying the data, and of particular interest is the delta-rs library \cite{DeltaioDeltars2024} that enables Python access to Delta Lake tables without using Spark. 
However, the delta-rs \cite{DeltaioDeltars2024} does not support \gls{HDFS}, and consequently \gls{HopsFS} \cite{niaziHopsFSScalingHierarchical2017}.

% Research Question
\subsection{Research Question}
\label{subsec:researchQuestion}
This research project has the ultimate objective to evaluate and compare the performance of the current Spark system that operates on Apache Hudi to a Rust system that uses delta-rs library \cite{DeltaioDeltars2024} operates on Delta Lake, using \gls{HopsFS} \cite{niaziHopsFSScalingHierarchical2017}. To achieve this, support for \gls{HDFS} (and thus also \gls{HopsFS}) must be added to the delta-rs library \cite{DeltaioDeltars2024} so that it can be compatible with the Hopsworks system. Thus, the project addresses the following two \glspl{RQ}:
\begin{enumerate}
    \item[RQ1:] How can we add support for \gls{HDFS} and \gls{HopsFS} to the delta-rs library?
    \item[RQ2:] What is the difference in latency and throughput between the current legacy system (Spark-based in writing) reading and writing to Apache Hudi compared to a delta-rs library-based reading and writing to Delta Lake, in \gls{HopsFS}?
\end{enumerate}

%Some text
% Small summary of the problem and formulation of a single research question

% \subsection{Scientific and engineering issues}
% % Outline of the various problems to address during implementation
% Delta-rs \cite{DeltaioDeltars2024}, as the name suggests, is a Rust \cite{RustProgrammingLanguage} library, that offers Python bindings. Rust is a compiled language, so it does not need an interpreter like Python or a virtual environment like Java. This means it is straightforward to embed and use Rust code as a library in another language, such as Python. 

% Currently, delta-rs does not support \gls{HDFS} and therefore, \gls{HopsFS}\cite{niaziHopsFSScalingHierarchical2017}. This means that adding \gls{HDFS} support for delta-rs becomes a requirement of this project. Additionally, it should be noted that to match the dependencies used in the repository, the object\_store \cite{Object_storeRust} interface of Apache DataFusion \cite{ApacheDataFusionApache} should be used.

\section{Purpose}
    \label{sec:purpose}
    In this section I will state the purpose of your thesis and the purpose of your degree project.\\
Describe who benefits and how they benefit if you achieve your goals. Include anticipated ethical, sustainability, social issues, etc. related to your project.


\section{Goals}
    \label{sec:goals}
    %%\sweExpl{Mål}
%%\sweExpl{Skilj på syfte och mål. Syftet är att åstakomma en förändring i något. Målen är vad som konkret skall göras för att om möjligt uppnå den önskade förändringen (syfte). }
%%
%%\generalExpl{State the goal/goals of this degree project.}
To accomplish the project's purpose and answer to the set \glspl{RQ}, a list of \glspl{G} is set, clarifying in a more step-by-step view what needs to be achieved in this work. 

\begin{enumerate}
    \item \glspl{G} aimed to answer RQ1: 
        \begin{enumerate}
            \item[G1:] Understand delta-rs library \cite{DeltaioDeltars2024} architecture and dependencies.
            \item[G2:] Identify what needs to be implemented to add \gls{HDFS} support to the delta-rs library \cite{DeltaioDeltars2024}.  
            \item[G3:] Implement \gls{HDFS} support in the delta-rs library \cite{DeltaioDeltars2024}.
        \end{enumerate}
    \item \glspl{G} aimed to answer RQ2:
        \begin{enumerate}
            \item[G4:] Design and choose an evaluation framework to evaluate the different read and write performances of the new Rust pipeline based on the delta-rs library \cite{DeltaioDeltars2024} and the old Apache Spark based pipeline.
            \item[G5:] Perform the experiments using the designed framework to understand if and how the two pipelines work at different data loads (from 10 GB to 1 TB).
        \end{enumerate}
\end{enumerate}

%%\generalExpl{In addition to presenting the goal(s), you might also state what the deliverables and results of the project are.}
Associated to these \glspl{G} a number of \glspl{D} will be created. 
\begin{enumerate}
    \item[D1:] Code implementation adding support to \gls{HDFS} in the delta-rs library. This \gls{D} is related to the completion of goals G1--G3. This deliverable also represent the system implementation contribution of the project.
    \item[D2:] Experiments results on the performance evaluation of the new Rust pipeline based on the delta-rs library \cite{DeltaioDeltars2024} compared to the old Apache Spark based pipeline.
    This \gls{D} is related to the completion of goals G4--G5.
    \item[D3:] This report, that on top of including the results, provides the whole research path that was followed, a clear background section and a results discussion.
\end{enumerate}

\section{Research Methodology}
    \label{sec:research_methodology}
    \sweExpl{Undersökningsmetod}
\sweExpl{Här anger du vilken vilken övergripande undersökningsstrategi eller metod du skall använda för att försöka besvara den akademiska frågeställning och samtidigt lösa det e v ursprungliga problemet. Ofta kan man använda ”lösandet av ursprungsproblemet” som en fallstudie kring en akademisk frågeställning. Du undersöker någon intressant fråga i ”skarpt” läge och samlar resultat och erfarenhet ur detta.\\
Tänk på att företaget ibland måste stå tillbaka i sin önskan och förväntan på projektets resultat till förmån för ny eller kompletterande ingenjörserfarenhet och vetenskap (ditt examensarbete). Det är du som student som bestämmer och löser fördelningen mellan dessa två intressen men se till att alla är informerade. }
\generalExpl{Introduce your choice of methodology/methodologies and method/methods – and the reason why you chose them. Contrast them with and explain why you did not choose other methodologies or methods. (The details of the actual methodology and method you have chosen will be given in Chapter~\ref{ch:method}. Note that in Chapter~\ref{ch:method}, the focus could be research strategies, data collection, data analysis, and quality assurance.)\\
In this section you should present your philosophical assumption(s), research method(s), and research approach(es).}

\section{Delimitations}
    \label{sec:delimitations}
    In this section I will describe the boundary/limits of your thesis project and what you are explicitly not going to do. This will help you bound your efforts - as you have clearly defined what is out of the scope of this thesis project. Explain the delimitations. These are all the things that could affect the study if they were examined and included in the degree project.

\section{Structure of the thesis}
    \label{sec:structure_thesis}
    This section will describe how the thesis is structured indicating various chapters.

\cleardoublepage

\chapter{Background}
    \label{ch:background}
    \input{contents/2-background/draft}

\cleardoublepage
\begin{comment}
\chapter{Method}
    \label{ch:method}
    \input{contents/3-method/draft}

\cleardoublepage

\chapter{Implementation}\engExpl{Choose your own chapter title to describe this}
    \label{ch:implementation}
    \input{contents/4-implementation/draft}

\cleardoublepage

\chapter{Results and Analysis}
    \label{ch:results_and_analysis}
    \input{contents/5-results_and_analysis/draft}

\cleardoublepage

\chapter{Conclusions and Future work}
    \label{ch:conclusions_and_future_work}
    \input{contents/6-conclusions_and_future_work/draft}
\end{comment}
%%%%%%%%%%%%%%%%%%%%%%%%%%%%%%%%%%%%%%%%%%%%%%%%%%%%%%%%%%%%%%%%%%%%%%%%%%%%%%%%%%%%%%%%
%%                                 REFERENCES
%%%%%%%%%%%%%%%%%%%%%%%%%%%%%%%%%%%%%%%%%%%%%%%%%%%%%%%%%%%%%%%%%%%%%%%%%%%%%%%%%%%%%%%%
\noindent\rule{\textwidth}{0.4mm}
%%\engExpl{In the references, let Zotero or other tool fill this in for you. I suggest an extended version of the IEEE style, to include URLs, DOIs, ISBNs, etc., to make it easier for your reader to find them. This will make life easier for your opponents and examiner. \\IEEE Editorial Style Manual: \url{https://www.ieee.org/content/dam/ieee-org/ieee/web/org/conferences/style_references_manual.pdf}}

\cleardoublepage

% Print the bibliography (and make it appear in the table of contents)
\renewcommand{\bibname}{References}

\ifbiblatex
    %\typeout{Biblatex current language is \currentlang}
    \printbibliography[heading=bibintoc]
\else
    \phantomsection  % make it include a hyperref - see https://tex.stackexchange.com/a/98995
    \addcontentsline{toc}{chapter}{References}
    \bibliography{references}
\fi

%%%%%%%%%%%%%%%%%%%%%%%%%%%%%%%%%%%%%%%%%%%%%%%%%%%%%%%%%%%%%%%%%%%%%%%%%%%%%%%%%%%%%%%%
%%                                 APPENDIX
%%%%%%%%%%%%%%%%%%%%%%%%%%%%%%%%%%%%%%%%%%%%%%%%%%%%%%%%%%%%%%%%%%%%%%%%%%%%%%%%%%%%%%%%
%% If you do not have an appendix, do not include the \textbackslash cleardoublepage command below; otherwise, the last page number in the metadata will be one too large.}
%\cleardoublepage

%\appendix
%\renewcommand{\chaptermark}[1]{\markboth{Appendix \thechapter\relax:\thinspace\relax#1}{}}

%\chapter{System architectures}
    \label{appx:sys_arch}
    % Small appendix expl.
This appendix reports the legacy architecture diagrams shown in Section \ref{sec:arch_sys} increased in size to improve readability.

\begin{figure}
    \begin{center}
      \includegraphics[angle=90,origin=c,keepaspectratio,height=12.5cm]{figures/2-background/FeatureStore-writing.png}
    \end{center}
    \caption[Legacy system - Write process - Magnified diagram]{Legacy system writing a Pandas DataFrame from a Python client to the Hopsworks offline Feature Store.  This image was magnified to enhance visualization.}
    \label{fig:appx_featurestore_writing}
\end{figure}

\begin{figure}
    \begin{center}
      \includegraphics[angle=90,origin=c,keepaspectratio,height=12.5cm]{figures/2-background/FeatureStore-reading.png}
    \end{center}
    \caption[Legacy system - Read process - Magnified diagram]{Legacy system reading a table from the Hopsworks offline feature store and loading it into the Python client's local memory. This image was magnified to enhance visualization.}
    \label{fig:appx_featurestore_reading}
\end{figure}

\chapter{Write experiments results}
    \label{appx:res_write}
    % Brief expl
This appendix reports all graphs and tables related to the write experiments conducted. Results are reported first expressed as latency (measured during the experiments) and then as throughput (computed from the latency and table size).

%%%%%%%%%%%%%%%%%%%%%%%%%%%%%%%%%%%%%%%%%%%%%%%%%%%%%%%%%%%%%%%%%
%%%%%%%%%%%%              LATENCY             %%%%%%%%%%%%%%%%%%%
%%%%%%%%%%%%%%%%%%%%%%%%%%%%%%%%%%%%%%%%%%%%%%%%%%%%%%%%%%%%%%%%%
\begin{figure}
    \centering
    \begin{minipage}[b]{\textwidth}
        \centering
        \captionof{table}[Write experiment - Latency - 1 CPU core]{Write experiment results expressed as latency. The experiment was performed with one \glstext{CPU} core.}
        \label{tbl:appx_res_write_time_1_core}
        \begin{tabular}{c r S[table-format=5.5] S[table-format=5.5] S[table-format=5.5]} 
            \toprule
            \multirow{2}{*}{{Pipeline\Tstrut\Bstrut}} & \multirow{2}{*}{{\thead{Number\\ of rows}}} & {\multirow{2}{*}{{\thead{Latency \\ (seconds)}}}} & \multicolumn{2}{c}{{\thead{Latency (seconds) \\95\% Confidence Interval}}}\\
                                                      &                                             &                                                   & {low} & {high}\\
            \midrule
            \multirow{5}{4em}{delta-rs\\ HopsFS} & 10K  &    1.25088 &    1.23807 &   1.26545\\ 
                                                 & 100K &    1.36828 &    1.33757 &   1.38982\\ 
                                                 & 1M   &    9.38152 &    9.23971 &   9.52904\\
                                                 & 6M   &   19.75469 &   19.33270 &  20.11785\\
                                                 & 60M  &  177.30707 &  174.62871 & 180.01732\\
            \midrule
            \multirow{5}{4em}{delta-rs\\ LocalFS} & 10K  &    0.03957 &   0.03770 &   0.04153\\ 
                                                  & 100K &    0.15240 &   0.14598 &   0.15888\\ 
                                                  & 1M   &    8.42252 &   8.28396 &   8.56376\\
                                                  & 6M   &   17.90634 &  17.48040 &  18.33585\\
                                                  & 60M  &  172.34552 & 169.74808 & 174.73138\\
            \midrule
            \multirow{5}{4em}{Legacy} & 10K  &    50.22767 &   49.53501 &   50.93664\\ 
                                      & 100K &    59.56187 &   58.89466 &   60.18496\\ 
                                      & 1M   &   112.19048 &  111.37162 &  113.00915\\
                                      & 6M   &   511.81693 &  510.75113 &  512.83672\\
                                      & 60M  &  2715.77285 & 2699.88061 & 2731.95225\\
            \bottomrule
        \end{tabular}
    \end{minipage}
    \begin{minipage}[b]{\textwidth}
        \centering
        \includegraphics[width=\textwidth]{figures/99-appendix/results-diagrams/write/write_time_1_core.png}
        \caption[Histogram of the write experiment - Latency - 1 CPU core]{Histogram in log-scale of the write experiment results expressed as latency. The experiment was performed with one \glstext{CPU} core.}
        \label{fig:appx_res_write_time_1_core}
    \end{minipage}
\end{figure}

\begin{figure}
    \centering
    \begin{minipage}[b]{\textwidth}
        \centering
        \captionof{table}[Write experiment - Latency - 2 CPU cores]{Write experiment results expressed as latency. The experiment was performed with two \glstext{CPU} cores.}
        \label{tbl:appx_res_write_time_2_cores}
        \begin{tabular}{c r S[table-format=5.5] S[table-format=5.5] S[table-format=5.5]} 
            \toprule
            \multirow{2}{*}{{Pipeline\Tstrut\Bstrut}} & \multirow{2}{*}{{\thead{Number\\ of rows}}} & {\multirow{2}{*}{{\thead{Latency \\ (seconds)}}}} & \multicolumn{2}{c}{{\thead{Latency (seconds) \\95\% Confidence Interval}}}\\
                                                      &                                             &                                                   & {low} & {high}\\
            \midrule
            \multirow{5}{4em}{delta-rs\\ HopsFS} & 10K  &    1.26239 &    1.25079 &   1.27639\\ 
                                                 & 100K &    1.30812 &    1.28050 &   1.33217\\ 
                                                 & 1M   &    8.51536 &    8.34333 &   8.70077\\
                                                 & 6M   &   16.29042 &   15.90659 &  16.67362\\
                                                 & 60M  &  134.06089 &  131.65031 & 136.39761\\
            \midrule
            \multirow{5}{4em}{delta-rs\\ LocalFS} & 10K  &    0.04823 &   0.04640 &   0.04997\\ 
                                                  & 100K &    0.13714 &   0.13402 &   0.14050\\ 
                                                  & 1M   &    7.18530 &   7.03747 &   7.35128\\
                                                  & 6M   &   15.26632 &  14.85172 &  15.65167\\
                                                  & 60M  &  129.82007 & 127.60020 & 132.04689\\
            \midrule
            \multirow{5}{4em}{Legacy} & 10K  &    50.72405 &   50.10769 &   51.30686\\ 
                                      & 100K &    59.78810 &   58.97997 &   60.47427\\ 
                                      & 1M   &   108.56499 &  108.01124 &  109.08128\\
                                      & 6M   &   473.37954 &  472.34534 &  474.43740\\
                                      & 60M  &  2340.77013 & 2333.99443 & 2347.97127\\
            \bottomrule
        \end{tabular}
    \end{minipage}
    \begin{minipage}[b]{\textwidth}
        \centering
        \includegraphics[width=\textwidth]{figures/99-appendix/results-diagrams/write/write_time_2_core.png}
        \caption[Histogram of the write experiment - Latency - 2 CPU cores]{Histogram in log-scale of the write experiment results expressed as latency. The experiment was performed with two \glstext{CPU} cores.}
        \label{fig:appx_res_write_time_2_cores}
    \end{minipage}
\end{figure}

\begin{figure}
    \centering
    \begin{minipage}[b]{\textwidth}
        \centering
        \captionof{table}[Write experiment - Latency - 4 CPU cores]{Write experiment results expressed as latency. The experiment was performed with four \glstext{CPU} cores.}
        \label{tbl:appx_res_write_time_4_cores}
        \begin{tabular}{c r S[table-format=5.5] S[table-format=5.5] S[table-format=5.5]} 
            \toprule
            \multirow{2}{*}{{Pipeline\Tstrut\Bstrut}} & \multirow{2}{*}{{\thead{Number\\ of rows}}} & {\multirow{2}{*}{{\thead{Latency \\ (seconds)}}}} & \multicolumn{2}{c}{{\thead{Latency (seconds) \\95\% Confidence Interval}}}\\
                                                      &                                             &                                                   & {low} & {high}\\
            \midrule
            \multirow{5}{4em}{delta-rs\\ HopsFS} & 10K  &    1.21642 &    1.20232 &   1.23231\\ 
                                                 & 100K &    1.33622 &    1.32294 &   1.34942\\ 
                                                 & 1M   &    8.41325 &    8.24770 &   8.58272\\
                                                 & 6M   &   16.22402 &   15.87946 &  16.59586\\
                                                 & 60M  &  124.10242 &  121.57723 & 126.81530\\
            \midrule
            \multirow{5}{4em}{delta-rs\\ LocalFS} & 10K  &    0.04572 &   0.04341 &   0.04807\\ 
                                                  & 100K &    0.13176 &   0.12880 &   0.13499\\ 
                                                  & 1M   &    7.18574 &   7.00679 &   7.36343\\
                                                  & 6M   &   14.55578 &  14.17679 &  14.94192\\
                                                  & 60M  &  121.37623 & 119.17256 & 123.69890\\
            \midrule
            \multirow{5}{4em}{Legacy} & 10K  &    51.28465 &   50.62282 &   51.90367\\ 
                                      & 100K &    59.52655 &   58.90537 &   60.15322\\ 
                                      & 1M   &   108.81674 &  108.25217 &  109.34234\\
                                      & 6M   &   481.98353 &  481.04435 &  482.92992\\
                                      & 60M  &  2346.04687 & 2336.99396 & 2355.19897\\
            \bottomrule
        \end{tabular}
    \end{minipage}
    \begin{minipage}[b]{\textwidth}
        \centering
        \includegraphics[width=\textwidth]{figures/99-appendix/results-diagrams/write/write_time_4_core.png}
        \caption[Histogram of the write experiment - Latency - 4 CPU cores]{Histogram in log-scale of the write experiment results expressed as latency. The experiment was performed with four \glstext{CPU} cores.}
        \label{fig:appx_res_write_time_4_cores}
    \end{minipage}
\end{figure}

\begin{figure}
    \centering
    \begin{minipage}[b]{\textwidth}
        \centering
        \captionof{table}[Write experiment - Latency - 8 CPU cores]{Write experiment results expressed as latency. The experiment was performed with eight \glstext{CPU} cores.}
        \label{tbl:appx_res_write_time_8_cores}
        \begin{tabular}{c r S[table-format=5.5] S[table-format=5.5] S[table-format=5.5]} 
            \toprule
            \multirow{2}{*}{{Pipeline\Tstrut\Bstrut}} & \multirow{2}{*}{{\thead{Number\\ of rows}}} & {\multirow{2}{*}{{\thead{Latency \\ (seconds)}}}} & \multicolumn{2}{c}{{\thead{Latency (seconds) \\95\% Confidence Interval}}}\\
                                                      &                                             &                                                   & {low} & {high}\\
            \midrule
            \multirow{5}{4em}{delta-rs\\ HopsFS} & 10K  &    1.36756 &    1.24934 &   1.57224\\ 
                                                 & 100K &    1.29243 &    1.26548 &   1.31099\\ 
                                                 & 1M   &    8.30120 &    8.14918 &   8.47040\\
                                                 & 6M   &   15.73847 &   15.28974 &  16.16084\\
                                                 & 60M  &  121.95014 &  119.59376 & 124.18097\\
            \midrule
            \multirow{5}{4em}{delta-rs\\ LocalFS} & 10K  &    0.04402 &   0.04174 &   0.04640\\ 
                                                  & 100K &    0.13648 &   0.13281 &   0.14061\\ 
                                                  & 1M   &    7.22872 &   7.07511 &   7.39893\\
                                                  & 6M   &   14.28157 &  13.90508 &  14.66126\\
                                                  & 60M  &  119.97915 & 117.76416 & 122.20882\\
            \midrule
            \multirow{5}{4em}{Legacy} & 10K  &    51.22859 &   50.59478 &   51.86476\\ 
                                      & 100K &    60.27751 &   59.72907 &   60.77130\\ 
                                      & 1M   &   109.38189 &  108.86830 &  109.88263\\
                                      & 6M   &   475.94345 &  474.83993 &  477.05274\\
                                      & 60M  &  2324.97917 & 2319.04203 & 2331.04794\\
            \bottomrule
        \end{tabular}
    \end{minipage}
    \begin{minipage}[b]{\textwidth}
        \centering
        \includegraphics[width=\textwidth]{figures/99-appendix/results-diagrams/write/write_time_8_core.png}
        \caption[Histogram of the write experiment - Latency - 8 CPU cores]{Histogram in log-scale of the write experiment results expressed as latency. The experiment was performed with eight \glstext{CPU} cores.}
        \label{fig:appx_res_write_time_8_cores}
    \end{minipage}
\end{figure}

%%%%%%%%%%%%%%%%%%%%%%%%%%%%%%%%%%%%%%%%%%%%%%%%%%%%%%%%%%%%%%%%%
%%%%%%%%%%%%             THROUGHPUT           %%%%%%%%%%%%%%%%%%%
%%%%%%%%%%%%%%%%%%%%%%%%%%%%%%%%%%%%%%%%%%%%%%%%%%%%%%%%%%%%%%%%%

\begin{figure}
    \centering
    \begin{minipage}[b]{\textwidth}
        \centering
        \captionof{table}[Write experiment - Throughput - 1 CPU core]{Write experiment results expressed as throughput. The experiment was performed with one \glstext{CPU} core.}
        \label{tbl:appx_res_write_throughput_1_core}
        \begin{tabular}{c r S[table-format=5.5] S[table-format=5.5] S[table-format=5.5]} 
            \toprule
            \multirow{2}{*}{{Pipeline\Tstrut\Bstrut}} & \multirow{2}{*}{{\thead{Number\\ of rows}}} & {\multirow{2}{*}{{\thead{Throughput \\ (k rows/second)}}}} & \multicolumn{2}{c}{{\thead{Throughput (k rows/second) \\95\% Confidence Interval}}}\\
                                                      &                                             &                                                          & {low} & {high}\\
            \midrule
            \multirow{5}{4em}{delta-rs\\ HopsFS} & 10K  &    7.99436 &    7.90230 &   8.07705\\ 
                                                 & 100K &    7.30843 &    7.19514 &   7.47621\\ 
                                                 & 1M   &  106.59242 &  104.94226 & 108.22850\\
                                                 & 6M   &  303.72533 &  298.24252 & 310.35491\\
                                                 & 60M  &  338.39598 &  333.30126 & 343.58610\\
            \midrule
            \multirow{5}{4em}{delta-rs\\ LocalFS} & 10K  &  252.68238 &  240.73405 &  265.18632\\ 
                                                  & 100K &  656.15739 &  629.36777 &  684.98518\\ 
                                                  & 1M   &  118.72919 &  116.77110 &  120.71514\\
                                                  & 6M   &  335.07675 &  327.22770 &  343.24143\\
                                                  & 60M  &  348.13784 &  343.38422 &  353.46496\\
            \midrule
            \multirow{5}{4em}{Legacy} & 10K  &     0.19909 &    0.19632 &    0.20187\\ 
                                      & 100K &     1.67892 &    1.66154 &    1.69794\\ 
                                      & 1M   &     8.91341 &    8.84884 &    8.97894\\
                                      & 6M   &    11.72294 &   11.69963 &   11.74740\\
                                      & 60M  &    22.09315 &   21.96231 &   22.22320\\
            \bottomrule
        \end{tabular}
    \end{minipage}
    \begin{minipage}[b]{\textwidth}
        \centering
        \includegraphics[width=\textwidth]{figures/99-appendix/results-diagrams/write/write_throughput_1_core.png}
        \caption[Histogram of the write experiment - Throughput - 1 CPU core]{Histogram in log-scale of the write experiment results expressed as throughput. The experiment was performed with one \glstext{CPU} core.}
        \label{fig:appx_res_write_throughput_1_core}
    \end{minipage}
\end{figure}

\begin{figure}
    \centering
    \begin{minipage}[b]{\textwidth}
        \centering
        \captionof{table}[Write experiment - Throughput - 2 CPU cores]{Write experiment results expressed as throughput. The experiment was performed with two \glstext{CPU} cores.}
        \label{tbl:appx_res_write_throughput_2_cores}
        \begin{tabular}{c r S[table-format=5.5] S[table-format=5.5] S[table-format=5.5]} 
            \toprule
            \multirow{2}{*}{{Pipeline\Tstrut\Bstrut}} & \multirow{2}{*}{{\thead{Number\\ of rows}}} & {\multirow{2}{*}{{\thead{Throughput \\ (k rows/second)}}}} & \multicolumn{2}{c}{{\thead{Throughput (k rows/second) \\95\% Confidence Interval}}}\\
                                                      &                                             &                                                          & {low} & {high}\\
            \midrule
            \multirow{5}{4em}{delta-rs\\ HopsFS} & 10K  &    7.92146 &    7.83458 &   7.99491\\ 
                                                 & 100K &   76.44507 &   75.06499 &  78.09433\\ 
                                                 & 1M   &  117.43478 &  114.93231 & 119.85614\\
                                                 & 6M   &  368.31440 &  359.84975 & 377.20198\\
                                                 & 60M  &  447.55780 &  439.89038 & 455.75281\\
            \midrule
            \multirow{5}{4em}{delta-rs\\ LocalFS} & 10K  &  207.31922 &  200.08626 &  215.49342\\ 
                                                  & 100K &  729.15967 &  711.73966 &  746.13854\\ 
                                                  & 1M   &  139.17297 &  136.03055 &  142.09647\\
                                                  & 6M   &  393.02185 &  383.34560 &  403.99352\\
                                                  & 60M  &  462.17814 &  454.38403 &  470.21868\\
            \midrule
            \multirow{5}{4em}{Legacy} & 10K  &     0.19714 &    0.19490 &    0.19957\\ 
                                      & 100K &     1.67257 &    1.65359 &    1.69549\\ 
                                      & 1M   &     9.21107 &    9.16747 &    9.25829\\
                                      & 6M   &    12.67481 &   12.64655 &   12.70257\\
                                      & 60M  &    25.63258 &   25.55397 &   25.70700\\
            \bottomrule
        \end{tabular}
    \end{minipage}
    \begin{minipage}[b]{\textwidth}
        \centering
        \includegraphics[width=\textwidth]{figures/99-appendix/results-diagrams/write/write_throughput_2_core.png}
        \caption[Histogram of the write experiment - Throughput - 2 CPU cores]{Histogram in log-scale of the write experiment results expressed as throughput. The experiment was performed with two \glstext{CPU} cores.}
        \label{fig:appx_res_write_throughput_2_cores}
    \end{minipage}
\end{figure}

\begin{figure}
    \centering
    \begin{minipage}[b]{\textwidth}
        \centering
        \captionof{table}[Write experiment - Throughput - 4 CPU cores]{Write experiment results expressed as throughput. The experiment was performed with four \glstext{CPU} cores.}
        \label{tbl:appx_res_write_throughput_4_cores}
        \begin{tabular}{c r S[table-format=5.5] S[table-format=5.5] S[table-format=5.5]} 
            \toprule
            \multirow{2}{*}{{Pipeline\Tstrut\Bstrut}} & \multirow{2}{*}{{\thead{Number\\ of rows}}} & {\multirow{2}{*}{{\thead{Throughput \\ (k rows/second)}}}} & \multicolumn{2}{c}{{\thead{Throughput (k rows/second) \\95\% Confidence Interval}}}\\
                                                      &                                             &                                                          & {low} & {high}\\
            \midrule
            \multirow{5}{4em}{delta-rs\\ HopsFS} & 10K  &    8.22083 &    8.11482 &   8.11482\\ 
                                                 & 100K &   74.83742 &   74.10566 &  75.58871\\ 
                                                 & 1M   &  118.86008 &  116.51310 & 121.24582\\
                                                 & 6M   &  369.82202 &  361.53577 & 377.84652\\
                                                 & 60M  &  483.47160 &  473.12899 & 493.51343\\
            \midrule
            \multirow{5}{4em}{delta-rs\\ LocalFS} & 10K  &  218.71364 &  208.02604 &  230.31412\\ 
                                                  & 100K &  758.92422 &  740.79474 &  776.39115\\ 
                                                  & 1M   &  139.16432 &  135.80613 &  142.71864\\
                                                  & 6M   &  412.20728 &  401.55459 &  423.22678\\
                                                  & 60M  &  494.33070 &  485.04875 &  503.47156\\
            \midrule
            \multirow{5}{4em}{Legacy} & 10K  &     0.19499 &    0.19266 &    0.19753\\ 
                                      & 100K &     1.67992 &    1.66242 &    1.69763\\ 
                                      & 1M   &     9.18976 &    9.14558 &    9.23768\\
                                      & 6M   &    12.44855 &   12.42416 &   12.47286\\
                                      & 60M  &    25.57493 &   25.47555 &   25.67400\\
            \bottomrule
        \end{tabular}
    \end{minipage}
    \begin{minipage}[b]{\textwidth}
        \centering
        \includegraphics[width=\textwidth]{figures/99-appendix/results-diagrams/write/write_throughput_4_core.png}
        \caption[Histogram of the write experiment - Throughput - 4 CPU cores]{Histogram in log-scale of the write experiment results expressed as throughput. The experiment was performed with four \glstext{CPU} cores.}
        \label{fig:appx_res_write_throughput_4_cores}
    \end{minipage}
\end{figure}

\begin{figure}
    \centering
    \begin{minipage}[b]{\textwidth}
        \centering
        \captionof{table}[Write experiment - Throughput - 8 CPU cores]{Write experiment results expressed as throughput. The experiment was performed with eight \glstext{CPU} cores.}
        \label{tbl:appx_res_write_throughput_8_cores}
        \begin{tabular}{c r S[table-format=5.5] S[table-format=5.5] S[table-format=5.5]} 
            \toprule
            \multirow{2}{*}{{Pipeline\Tstrut\Bstrut}} & \multirow{2}{*}{{\thead{Number\\ of rows}}} & {\multirow{2}{*}{{\thead{Throughput \\ (k rows/second)}}}} & \multicolumn{2}{c}{{\thead{Throughput (k rows/second) \\95\% Confidence Interval}}}\\
                                                      &                                             &                                                          & {low} & {high}\\
            \midrule
            \multirow{5}{4em}{delta-rs\\ HopsFS} & 10K  &    7.31228 &    6.36032 &    8.00422\\ 
                                                 & 100K &   77.37337 &   76.27782 &   79.02104\\ 
                                                 & 1M   &  120.46439 &  118.05814 &  122.71160\\
                                                 & 6M   &  381.23126 &  371.26772 &  392.41978\\
                                                 & 60M  &  492.00431 &  483.16579 &  501.69837\\
            \midrule
            \multirow{5}{4em}{delta-rs\\ LocalFS} & 10K  &  227.12095 &  215.48714 &  239.54128\\ 
                                                  & 100K &  732.70141 &  711.16038 &  752.93200\\ 
                                                  & 1M   &  138.33701 &  135.15466 &  141.34041\\
                                                  & 6M   &  420.12165 &  409.24176 &  431.49669\\
                                                  & 60M  &  500.08688 &  490.96288 &  509.49286\\
            \midrule
            \multirow{5}{4em}{Legacy} & 10K  &     0.19520 &    0.19280 &    0.19764\\ 
                                      & 100K &     1.65899 &    1.64551 &    1.67422\\ 
                                      & 1M   &     9.14228 &    9.10061 &    9.18540\\
                                      & 6M   &    12.60653 &   12.57722 &   12.63583\\
                                      & 60M  &    25.80668 &   25.73949 &   25.87275\\
            \bottomrule
        \end{tabular}
    \end{minipage}
    \begin{minipage}[b]{\textwidth}
        \centering
        \includegraphics[width=\textwidth]{figures/99-appendix/results-diagrams/write/write_throughput_8_core.png}
        \caption[Histogram of the write experiment - Throughput - 8 CPU cores]{Histogram in log-scale of the write experiment results expressed as throughput. The experiment was performed with eight \glstext{CPU} cores.}
        \label{fig:appx_res_write_throughput_8_cores}
    \end{minipage}
\end{figure}

\chapter{Read experiments results}
    \label{appx:res_read}
    % Brief expl
This appendix reports all graphs and tables related to the read experiments conducted. Results are reported first expressed as latency (measured during the experiments) and then as throughput (computed from the latency and table size).


%%%%%%%%%%%%%%%%%%%%%%%%%%%%%%%%%%%%%%%%%%%%%%%%%%%%%%%%%%%%%%%%%
%%%%%%%%%%%%              LATENCY             %%%%%%%%%%%%%%%%%%%
%%%%%%%%%%%%%%%%%%%%%%%%%%%%%%%%%%%%%%%%%%%%%%%%%%%%%%%%%%%%%%%%%
\begin{figure}
    \centering
    \begin{minipage}[b]{\textwidth}
        \centering
        \captionof{table}[Read experiment - Latency - 1 CPU core]{Read experiment results expressed as latency. The experiment was performed with one \glstext{CPU} core.}
        \label{tbl:appx_res_read_time_1_core}
        \begin{tabular}{c r S[table-format=5.5] S[table-format=5.5] S[table-format=5.5]} 
            \toprule
            \multirow{2}{*}{{Pipeline\Tstrut\Bstrut}} & \multirow{2}{*}{{\thead{Number\\ of rows}}} & {\multirow{2}{*}{{\thead{Latency \\ (seconds)}}}} & \multicolumn{2}{c}{{\thead{Latency (seconds) \\95\% Confidence Interval}}}\\
                                                      &                                             &                                                   & {low} & {high}\\
            \midrule
            \multirow{5}{4em}{delta-rs\\ HopsFS} & 10K  &    0.05342 &    0.03916 &    0.08112\\ 
                                                 & 100K &    0.05757 &    0.05518 &    0.06046\\ 
                                                 & 1M   &    0.53855 &    0.52558 &    0.55229\\
                                                 & 6M   &    1.94899 &    1.93007 &    1.96860\\
                                                 & 60M  &   22.98065 &   22.84067 &   23.14206\\
            \midrule
            \multirow{5}{4em}{delta-rs\\ LocalFS} & 10K  &    0.00419 &    0.00268 &    0.00644\\ 
                                                  & 100K &    0.02696 &    0.01966 &    0.03433\\ 
                                                  & 1M   &    0.42009 &    0.40613 &    0.43563\\
                                                  & 6M   &    1.68223 &    1.65981 &    1.70440\\
                                                  & 60M  &   19.56547 &   19.34690 &   19.77724\\
            \midrule
            \multirow{5}{4em}{Legacy} & 10K  &     0.63159 &    0.62414 &    0.64157\\ 
                                      & 100K &     2.65010 &    2.64272 &    2.65876\\ 
                                      & 1M   &     8.59636 &    8.34094 &    8.90047\\
                                      & 6M   &    33.52964 &   33.23886 &   33.86591\\
                                      & 60M  &    33.69772 &   33.36262 &   34.08665\\
            \bottomrule
        \end{tabular}
    \end{minipage}
    \begin{minipage}[b]{\textwidth}
        \centering
        \includegraphics[width=\textwidth]{figures/99-appendix/results-diagrams/read/read_time_1_core.png}
        \caption[Histogram of the read experiment - Latency - 1 CPU core]{Histogram in log-scale of the read experiment results expressed as latency. The experiment was performed with one \glstext{CPU} core.}
        \label{fig:appx_res_read_time_1_core}
    \end{minipage}
\end{figure}

\begin{figure}
    \centering
    \begin{minipage}[b]{\textwidth}
        \centering
        \captionof{table}[Read experiment - Latency - 2 CPU cores]{Read experiment results expressed as latency. The experiment was performed with two \glstext{CPU} cores.}
        \label{tbl:appx_res_read_time_2_cores}
        \begin{tabular}{c r S[table-format=5.5] S[table-format=5.5] S[table-format=5.5]} 
            \toprule
            \multirow{2}{*}{{Pipeline\Tstrut\Bstrut}} & \multirow{2}{*}{{\thead{Number\\ of rows}}} & {\multirow{2}{*}{{\thead{Latency \\ (seconds)}}}} & \multicolumn{2}{c}{{\thead{Latency (seconds) \\95\% Confidence Interval}}}\\
                                                      &                                             &                                                   & {low} & {high}\\
            \midrule
            \multirow{5}{4em}{delta-rs\\ HopsFS} & 10K  &    0.04132 &    0.03933 &    0.04378\\ 
                                                 & 100K &    0.05690 &    0.05123 &    0.06693\\ 
                                                 & 1M   &    0.23413 &    0.22528 &    0.24426\\
                                                 & 6M   &    0.90832 &    0.89967 &    0.91744\\
                                                 & 60M  &   11.41325 &   11.27661 &   11.58899\\
            \midrule
            \multirow{5}{4em}{delta-rs\\ LocalFS} & 10K  &    0.00287 &    0.00278 &    0.00299\\ 
                                                  & 100K &    0.01306 &    0.01041 &    0.01610\\ 
                                                  & 1M   &    0.19977 &    0.18858 &    0.21056\\
                                                  & 6M   &    0.74764 &    0.73503 &    0.76013\\
                                                  & 60M  &    9.44693 &    9.37207 &    9.51753\\
            \midrule
            \multirow{5}{4em}{Legacy} & 10K  &     0.62492 &    0.62210 &    0.62822\\ 
                                      & 100K &     2.66339 &    2.65616 &    2.67166\\ 
                                      & 1M   &     8.61667 &    8.30989 &    8.94938\\
                                      & 6M   &    33.37519 &   33.09688 &   33.67065\\
                                      & 60M  &    33.64281 &   33.30150 &   34.06307\\
            \bottomrule
        \end{tabular}
    \end{minipage}
    \begin{minipage}[b]{\textwidth}
        \centering
        \includegraphics[width=\textwidth]{figures/99-appendix/results-diagrams/read/read_time_2_core.png}
        \caption[Histogram of the read experiment - Latency - 2 CPU cores]{Histogram in log-scale of the read experiment results expressed as latency. The experiment was performed with two \glstext{CPU} cores.}
        \label{fig:appx_res_read_time_2_cores}
    \end{minipage}
\end{figure}

\begin{figure}
    \centering
    \begin{minipage}[b]{\textwidth}
        \centering
        \captionof{table}[Read experiment - Latency - 4 CPU cores]{Read experiment results expressed as latency. The experiment was performed with four \glstext{CPU} cores.}
        \label{tbl:appx_res_read_time_4_cores}
        \begin{tabular}{c r S[table-format=5.5] S[table-format=5.5] S[table-format=5.5]} 
            \toprule
            \multirow{2}{*}{{Pipeline\Tstrut\Bstrut}} & \multirow{2}{*}{{\thead{Number\\ of rows}}} & {\multirow{2}{*}{{\thead{Latency \\ (seconds)}}}} & \multicolumn{2}{c}{{\thead{Latency (seconds) \\95\% Confidence Interval}}}\\
                                                      &                                             &                                                   & {low} & {high}\\
            \midrule
            \multirow{5}{4em}{delta-rs\\ HopsFS} & 10K  &    0.04336 &    0.03922 &   0.05092\\ 
                                                 & 100K &    0.05540 &    0.05378 &   0.05789\\ 
                                                 & 1M   &    0.18847 &    0.15157 &   0.25189\\
                                                 & 6M   &    0.53124 &    0.50778 &   0.57105\\
                                                 & 60M  &    5.58011 &    5.54397 &   5.61936\\
            \midrule
            \multirow{5}{4em}{delta-rs\\ LocalFS} & 10K  &    0.00268 &   0.00259 &   0.00279\\ 
                                                  & 100K &    0.00923 &   0.00852 &   0.01020\\ 
                                                  & 1M   &    0.08971 &   0.08388 &   0.09503\\
                                                  & 6M   &    0.37021 &   0.36018 &   0.38032\\
                                                  & 60M  &    4.81023 &   4.79338 &   4.82789\\
            \midrule
            \multirow{5}{4em}{Legacy} & 10K  &     0.63583 &    0.62352 &    0.65908\\ 
                                      & 100K &     2.63985 &    2.63349 &    2.64623\\ 
                                      & 1M   &     8.75238 &    8.50725 &    9.01383\\
                                      & 6M   &    33.45286 &   33.19461 &   33.75646\\
                                      & 60M  &    33.65245 &   33.27016 &   34.03900\\
            \bottomrule
        \end{tabular}
    \end{minipage}
    \begin{minipage}[b]{\textwidth}
        \centering
        \includegraphics[width=\textwidth]{figures/99-appendix/results-diagrams/read/read_time_4_core.png}
        \caption[Histogram of the read experiment - Latency - 4 CPU cores]{Histogram in log-scale of the read experiment results expressed as latency. The experiment was performed with four \glstext{CPU} cores.}
        \label{fig:appx_res_read_time_4_cores}
    \end{minipage}
\end{figure}

\begin{figure}
    \centering
    \begin{minipage}[b]{\textwidth}
        \centering
        \captionof{table}[Read experiment - Latency - 8 CPU cores]{Read experiment results expressed as latency. The experiment was performed with eight \glstext{CPU} cores.}
        \label{tbl:appx_res_read_time_8_cores}
        \begin{tabular}{c r S[table-format=5.5] S[table-format=5.5] S[table-format=5.5]} 
            \toprule
            \multirow{2}{*}{{Pipeline\Tstrut\Bstrut}} & \multirow{2}{*}{{\thead{Number\\ of rows}}} & {\multirow{2}{*}{{\thead{Latency \\ (seconds)}}}} & \multicolumn{2}{c}{{\thead{Latency (seconds) \\95\% Confidence Interval}}}\\
                                                      &                                             &                                                   & {low} & {high}\\
            \midrule
            \multirow{5}{4em}{delta-rs\\ HopsFS} & 10K  &    0.04330 &    0.03885 &    0.05119\\ 
                                                 & 100K &    0.05458 &    0.05286 &    0.05663\\ 
                                                 & 1M   &    0.17390 &    0.16992 &    0.17808\\
                                                 & 6M   &    0.49729 &    0.48655 &    0.51019\\
                                                 & 60M  &    2.94236 &    2.85554 &    3.06360\\
            \midrule
            \multirow{5}{4em}{delta-rs\\ LocalFS} & 10K  &    0.00294 &    0.00284 &    0.00307\\ 
                                                  & 100K &    0.00948 &    0.00872 &    0.01056\\ 
                                                  & 1M   &    0.04308 &    0.03934 &    0.04830\\
                                                  & 6M   &    0.17548 &    0.17009 &    0.18080\\
                                                  & 60M  &    2.28550 &    2.27501 &    2.29607\\
            \midrule
            \multirow{5}{4em}{Legacy} & 10K  &     0.62739 &    0.62245 &    0.63259\\ 
                                      & 100K &     2.66217 &    2.65309 &    2.67218\\ 
                                      & 1M   &     8.34757 &    8.13471 &    8.60942\\
                                      & 6M   &    33.42815 &   33.15376 &   33.74947\\
                                      & 60M  &    33.14341 &   32.88303 &   33.41299\\
            \bottomrule
        \end{tabular}
    \end{minipage}
    \begin{minipage}[b]{\textwidth}
        \centering
        \includegraphics[width=\textwidth]{figures/99-appendix/results-diagrams/read/read_time_8_core.png}
        \caption[Histogram of the read experiment - Latency - 8 CPU cores]{Histogram in log-scale of the read experiment results expressed as latency. The experiment was performed with eight \glstext{CPU} cores.}
        \label{fig:appx_res_read_time_8_cores}
    \end{minipage}
\end{figure}

%%%%%%%%%%%%%%%%%%%%%%%%%%%%%%%%%%%%%%%%%%%%%%%%%%%%%%%%%%%%%%%%%
%%%%%%%%%%%%             THROUGHPUT           %%%%%%%%%%%%%%%%%%%
%%%%%%%%%%%%%%%%%%%%%%%%%%%%%%%%%%%%%%%%%%%%%%%%%%%%%%%%%%%%%%%%%

\begin{figure}
    \centering
    \begin{minipage}[b]{\textwidth}
        \centering
        \captionof{table}[Read experiment - Throughput - 1 CPU core]{Read experiment results expressed as throughput. The experiment was performed with one \glstext{CPU} core.}
        \label{tbl:appx_res_read_throughput_1_core}
        \begin{tabular}{c r S[table-format=5.5] S[table-format=5.5] S[table-format=5.5]} 
            \toprule
            \multirow{2}{*}{{Pipeline\Tstrut\Bstrut}} & \multirow{2}{*}{{\thead{Number\\ of rows}}} & {\multirow{2}{*}{{\thead{Throughput \\ (k rows/second)}}}} & \multicolumn{2}{c}{{\thead{Throughput (k rows/second) \\95\% Confidence Interval}}}\\
                                                      &                                             &                                                          & {low} & {high}\\
            \midrule
            \multirow{5}{4em}{delta-rs\\ HopsFS} & 10K  &  187.16853 &  123.26555 &  255.36173\\ 
                                                 & 100K & 1736.90799 & 1653.91940 & 1811.92857\\ 
                                                 & 1M   & 1856.83167 & 1810.61318 & 1902.65361\\
                                                 & 6M   & 3078.51299 & 3047.83914 & 3108.69431\\
                                                 & 60M  & 2610.89146 & 2592.68185 & 2626.89240\\
            \midrule
            \multirow{5}{4em}{delta-rs\\ LocalFS} & 10K  & 2384.58699 & 1552.08552 & 3721.36068\\ 
                                                  & 100K & 3708.25787 & 2912.43600 & 5084.75715\\ 
                                                  & 1M   & 2380.40381 & 2295.50154 & 2462.25814\\
                                                  & 6M   & 3566.67454 & 3520.29796 & 3614.87006\\
                                                  & 60M  & 3066.62644 & 3033.78996 & 3101.27165\\
            \midrule
            \multirow{5}{4em}{Legacy} & 10K  &    15.83285 &   15.58654 &   16.02196\\ 
                                      & 100K &    37.73432 &   37.61140 &   37.83975\\ 
                                      & 1M   &   116.32820 &  112.35350 &  119.89044\\
                                      & 6M   &   178.94612 &  177.16928 &  180.51159\\
                                      & 60M  &  1780.53563 & 1760.21974 & 1798.41962\\
            \bottomrule
        \end{tabular}
    \end{minipage}
    \begin{minipage}[b]{\textwidth}
        \centering
        \includegraphics[width=\textwidth]{figures/99-appendix/results-diagrams/read/read_throughput_1_core.png}
        \caption[Histogram of the read experiment - Throughput - 1 CPU core]{Histogram in log-scale of the read experiment results expressed as throughput. The experiment was performed with one \glstext{CPU} core.}
        \label{fig:appx_res_read_throughput_1_core}
    \end{minipage}
\end{figure}

\begin{figure}
    \centering
    \begin{minipage}[b]{\textwidth}
        \centering
        \captionof{table}[Read experiment - Throughput - 2 CPU cores]{Read experiment results expressed as throughput. The experiment was performed with two \glstext{CPU} cores.}
        \label{tbl:appx_res_read_throughput_2_cores}
        \begin{tabular}{c r S[table-format=5.5] S[table-format=5.5] S[table-format=5.5]} 
            \toprule
            \multirow{2}{*}{{Pipeline\Tstrut\Bstrut}} & \multirow{2}{*}{{\thead{Number\\ of rows}}} & {\multirow{2}{*}{{\thead{Throughput \\ (k rows/second)}}}} & \multicolumn{2}{c}{{\thead{Throughput (k rows/second) \\95\% Confidence Interval}}}\\
                                                      &                                             &                                                          & {low} & {high}\\
            \midrule
            \multirow{5}{4em}{delta-rs\\ HopsFS} & 10K  &  241.97833 &  228.37018 &  254.22419\\ 
                                                 & 100K & 1757.17930 & 1494.01876 & 1951.81229\\ 
                                                 & 1M   & 4271.12139 & 4093.98917 & 4438.84612\\
                                                 & 6M   & 6605.54323 & 6539.93631 & 6669.08756\\
                                                 & 60M  & 5257.04658 & 5177.32395 & 5320.74334\\
            \midrule
            \multirow{5}{4em}{delta-rs\\ LocalFS} & 10K  & 3479.95285 & 3339.11982 & 3592.42255\\ 
                                                  & 100K & 7652.74709 & 6210.29638 & 9604.33293\\ 
                                                  & 1M   & 5005.61642 & 4749.17943 & 5302.53656\\
                                                  & 6M   & 8025.19634 & 7893.33251 & 8162.84493\\
                                                  & 60M  & 6351.26715 & 6304.15538 & 6401.99467\\
            \midrule
            \multirow{5}{4em}{Legacy} & 10K  &   16.00183 &   15.91773 &   16.07453\\ 
                                      & 100K &   37.54607 &   37.42979 &   37.64822\\ 
                                      & 1M   &  116.05404 &  111.73951 &  120.33848\\
                                      & 6M   &  179.77424 &  178.19671 &  181.28593\\
                                      & 60M  & 1783.44198 & 1761.43830 & 1801.72013\\
            \bottomrule
        \end{tabular}
    \end{minipage}
    \begin{minipage}[b]{\textwidth}
        \centering
        \includegraphics[width=\textwidth]{figures/99-appendix/results-diagrams/read/read_throughput_2_core.png}
        \caption[Histogram of the read experiment - Throughput - 2 CPU cores]{Histogram in log-scale of the read experiment results expressed as throughput. The experiment was performed with two \glstext{CPU} cores.}
        \label{fig:appx_res_read_throughput_2_cores}
    \end{minipage}
\end{figure}

\begin{figure}
    \centering
    \begin{minipage}[b]{\textwidth}
        \centering
        \captionof{table}[Read experiment - Throughput - 4 CPU cores]{Read experiment results expressed as throughput. The experiment was performed with four \glstext{CPU} cores.}
        \label{tbl:appx_res_read_throughput_4_cores}
        \begin{tabular}{c r S[table-format=5.5] S[table-format=5.5] S[table-format=5.5]} 
            \toprule
            \multirow{2}{*}{{Pipeline\Tstrut\Bstrut}} & \multirow{2}{*}{{\thead{Number\\ of rows}}} & {\multirow{2}{*}{{\thead{Throughput \\ (k rows/second)}}}} & \multicolumn{2}{c}{{\thead{Throughput (k rows/second) \\95\% Confidence Interval}}}\\
                                                      &                                             &                                                          & {low} & {high}\\
            \midrule
            \multirow{5}{4em}{delta-rs\\ HopsFS} & 10K  &   230.61925 &   196.37366 &   254.93232\\ 
                                                 & 100K &  1804.73106 &  1727.27256 &  1859.40719\\ 
                                                 & 1M   &  5305.74239 &  3969.95699 &  6597.37404\\
                                                 & 6M   & 11294.12619 & 10506.91939 & 11816.03313\\
                                                 & 60M  & 10752.47511 & 10677.35647 & 10822.55620\\
            \midrule
            \multirow{5}{4em}{delta-rs\\ LocalFS} & 10K  &  3720.94104 &  3572.45085 &  3854.74973\\ 
                                                  & 100K & 10830.88457 &  9802.96062 & 11735.92863\\ 
                                                  & 1M   & 11146.06743 & 10522.54071 & 11921.62257\\
                                                  & 6M   & 16206.97330 & 15775.94387 & 16657.93725\\
                                                  & 60M  & 12473.40492 & 12427.78728 & 12517.25967\\
            \midrule
            \multirow{5}{4em}{Legacy} & 10K  &    15.72724 &   15.17259 &   16.03790\\ 
                                      & 100K &    37.88093 &   37.78959 &   37.97237\\ 
                                      & 1M   &   114.25460 &  110.94061 &  117.54671\\
                                      & 6M   &   179.35680 &  177.74372 &  180.75222\\
                                      & 60M  &  1782.93073 & 1762.68396 & 1803.41764\\
            \bottomrule
        \end{tabular}
    \end{minipage}
    \begin{minipage}[b]{\textwidth}
        \centering
        \includegraphics[width=\textwidth]{figures/99-appendix/results-diagrams/read/read_throughput_4_core.png}
        \caption[Histogram of the read experiment - Throughput - 4 CPU cores]{Histogram in log-scale of the read experiment results expressed as throughput. The experiment was performed with four \glstext{CPU} cores.}
        \label{fig:appx_res_read_throughput_4_cores}
    \end{minipage}
\end{figure}

\begin{figure}
    \centering
    \begin{minipage}[b]{\textwidth}
        \centering
        \captionof{table}[Read experiment - Throughput - 8 CPU cores]{Read experiment results expressed as throughput. The experiment was performed with eight \glstext{CPU} cores.}
        \label{tbl:appx_res_read_throughput_8_cores}
        \begin{tabular}{c r S[table-format=5.5] S[table-format=5.5] S[table-format=5.5]} 
            \toprule
            \multirow{2}{*}{{Pipeline\Tstrut\Bstrut}} & \multirow{2}{*}{{\thead{Number\\ of rows}}} & {\multirow{2}{*}{{\thead{Throughput \\ (k rows/second)}}}} & \multicolumn{2}{c}{{\thead{Throughput (k rows/second) \\95\% Confidence Interval}}}\\
                                                      &                                             &                                                          & {low} & {high}\\
            \midrule
            \multirow{5}{4em}{delta-rs\\ HopsFS} & 10K  &   230.92518 &   195.34544 &   257.36067\\ 
                                                 & 100K &  1832.05683 &  1765.75684 &  1891.54456\\ 
                                                 & 1M   &  5750.34366 &  5615.17071 &  5885.01795\\
                                                 & 6M   & 12065.18202 & 11760.17893 & 12331.70977\\
                                                 & 60M  & 20391.77956 & 19584.75363 & 21011.72136\\
            \midrule
            \multirow{5}{4em}{delta-rs\\ LocalFS} & 10K  &  3390.32242 &  3256.07486 &  3510.69231\\ 
                                                  & 100K & 10545.41087 &  9465.27536 & 11463.06073\\ 
                                                  & 1M   & 23212.46679 & 20701.23033 & 25414.13122\\
                                                  & 6M   & 34191.39637 & 33185.18129 & 35275.25456\\
                                                  & 60M  & 26252.42019 & 26131.51836 & 26373.48880\\
            \midrule
            \multirow{5}{4em}{Legacy} & 10K  &    15.93901 &   15.80791 &   16.06539\\ 
                                      & 100K &    37.56330 &   37.42250 &   37.69186\\ 
                                      & 1M   &   119.79520 &  116.15177 &  122.92995\\
                                      & 6M   &   179.48942 &  177.78054 &  180.97490\\
                                      & 60M  &  1810.31436 & 1795.70867 & 1824.64883\\
            \bottomrule
        \end{tabular}
    \end{minipage}
    \begin{minipage}[b]{\textwidth}
        \centering
        \includegraphics[width=\textwidth]{figures/99-appendix/results-diagrams/read/read_throughput_8_core.png}
        \caption[Histogram of the read experiment - Throughput - 8 CPU cores]{Histogram in log-scale of the read experiment results expressed as throughput. The experiment was performed with eight \glstext{CPU} cores.}
        \label{fig:appx_res_read_throughput_8_cores}
    \end{minipage}
\end{figure}

\chapter{Legacy pipeline write latency breakdown results}
    \label{appx:res_hudi}
    % Brief expl
This appendix reports all graphs and tables related to all write latency breakdown of the upload and materialization steps in the legacy pipeline.

\begin{figure}
    \centering
    \begin{minipage}[b]{\textwidth}
        \centering
        \captionof{table}{Contributions to the write latency of the upload and materialization steps in the legacy pipeline performed with 1 \glstext{CPU} core}
        \label{tbl:appx_hudi_virtualiz_breakdown_1_core}
        \begin{tabular}{l r S[table-format=5.4] S[table-format=5.4] S[table-format=5.4] S[table-format=5.4]} 
            \toprule
            {\multirow{2}{*}{Step\Tstrut\Bstrut}} & \multirow{2}{*}{{\thead{Number\\ of rows}}} & {\multirow{2}{*}{{\thead{Latency \\ (seconds)}}}} & \multicolumn{2}{c}{{\thead{Latency (seconds) \\95\% Confidence Interval}}}\\
                                    &                                             &                                                   & {low} & {high}                                                            \\
            \midrule
            upload                  & \multirow{2}{*}{10K}                        &    2.4865                                         &    2.3896 &    2.6261                                                      \\ 
            materialize           &                                             &   47.7262                                         &   47.0445 &   48.4031                                                      \\
            \midrule
            upload                  & \multirow{2}{*}{100K}                       &    3.6684                                         &    3.6310 &    3.7098                                                      \\                                                                 
            materialize            &                                             &   55.9005                                         &   55.2494 &   56.5541                                                      \\
            \midrule
            upload                  & \multirow{2}{*}{1M}                         &   22.5934                                         &   22.4496 &   22.7349                                                      \\                                                                 
            materialize             &                                             &   89.5754                                         &   88.8286 &   90.3049                                                      \\
            \midrule
            upload                 & \multirow{2}{*}{6M}                         &  244.6123                                         &  244.0368 &  245.1905                                                      \\                                                                 
            materialize             &                                             &  267.2490                                         &  266.4287 &  268.1549                                                      \\
            \midrule
            upload                  & \multirow{2}{*}{6M}                         & 2437.7840                                         & 2422.8704 & 2453.6746                                                      \\                                                                 
            materialize             &                                             &  278.0504                                         &  276.2340 &  280.0921                                                      \\
            \bottomrule
        \end{tabular}
    \end{minipage}
    \begin{minipage}[b]{\textwidth}
        \centering
        \includegraphics[width=\textwidth]{figures/99-appendix/results-diagrams/write/hudi_upload_materialize/hudi_virtualiz_1_core.png}
        \caption{Histogram in log-scale displaying the contributions to the write latency of the upload and materialization steps in the legacy pipeline performed with 1 \glstext{CPU} core}
        \label{fig:appx_hudi_virtualiz_breakdown_1_core}
    \end{minipage}
\end{figure}

\begin{figure}
    \centering
    \begin{minipage}[b]{\textwidth}
        \centering
        \captionof{table}{Contributions to the write latency of the upload and materialization steps in the legacy pipeline performed with 2 \glstext{CPU} cores}
        \label{tbl:appx_hudi_virtualiz_breakdown_2_cores}
        \begin{tabular}{l r S[table-format=5.4] S[table-format=5.4] S[table-format=5.4] S[table-format=5.4]} 
            \toprule
            {\multirow{2}{*}{Step\Tstrut\Bstrut}} & \multirow{2}{*}{{\thead{Number\\ of rows}}} & {\multirow{2}{*}{{\thead{Latency \\ (seconds)}}}} & \multicolumn{2}{c}{{\thead{Latency (seconds) \\95\% Confidence Interval}}}\\
                                    &                                             &                                                   & {low} & {high}                                                            \\
            \midrule
            upload                  & \multirow{2}{*}{10K}                        &    2.3873                                         &    2.3276 &    2.4466                                                      \\ 
            materialize             &                                             &   48.3305                                         &   47.7020 &   48.9923                                                      \\
            \midrule
            upload                  & \multirow{2}{*}{100K}                       &    3.4348                                         &    3.4008 &    3.4671                                                      \\                                                                 
            materialize             &                                             &   56.3367                                         &   55.5626 &   57.1129                                                      \\
            \midrule
            upload                  & \multirow{2}{*}{1M}                         &   18.6349                                         &   18.5673 &   18.7104                                                      \\                                                                 
            materialize             &                                             &   89.9267                                         &   89.4012 &   90.4514                                                      \\
            \midrule
            upload                  & \multirow{2}{*}{6M}                         &  205.8854                                         &  205.2177 &  206.4984                                                      \\                                                                 
            materialize             &                                             &  267.5079                                         &  266.6853 &  268.3512                                                      \\
            \midrule
            upload                  & \multirow{2}{*}{6M}                         & 2064.1357                                         & 2057.6396 & 2070.4450                                                      \\                                                                 
            materialize             &                                             &  276.9608                                         &  275.7156 &  278.2849                                                      \\
            \bottomrule
        \end{tabular}
    \end{minipage}
    \begin{minipage}[b]{\textwidth}
        \centering
        \includegraphics[width=\textwidth]{figures/99-appendix/results-diagrams/write/hudi_upload_materialize/hudi_virtualiz_2_core.png}
        \caption{Histogram in log-scale displaying the contributions to the write latency of the upload and materialization steps in the legacy pipeline performed with 2 \glstext{CPU} cores}
        \label{fig:appx_hudi_virtualiz_breakdown_2_core}
    \end{minipage}
\end{figure}

\begin{figure}
    \centering
    \begin{minipage}[b]{\textwidth}
        \centering
        \captionof{table}{Contributions to the write latency of the upload and materialization steps in the legacy pipeline performed with 4 \glstext{CPU} cores}
        \label{tbl:appx_hudi_virtualiz_breakdown_4_cores}
        \begin{tabular}{l r S[table-format=5.4] S[table-format=5.4] S[table-format=5.4] S[table-format=5.4]} 
            \toprule
            {\multirow{2}{*}{Step\Tstrut\Bstrut}} & \multirow{2}{*}{{\thead{Number\\ of rows}}} & {\multirow{2}{*}{{\thead{Latency \\ (seconds)}}}} & \multicolumn{2}{c}{{\thead{Latency (seconds) \\95\% Confidence Interval}}}\\
                                    &                                             &                                                   & {low} & {high}                                                            \\
            \midrule
            upload                  & \multirow{2}{*}{10K}                        &    2.3846                                         &    2.3299 &    2.4335                                                      \\ 
            materialize             &                                             &   48.9061                                         &   48.2436 &   49.5470                                                      \\
            \midrule
            upload                  & \multirow{2}{*}{100K}                       &    3.4650                                         &    3.4245 &    3.5071                                                      \\                                                                 
            materialize             &                                             &   56.0524                                         &   55.3682 &   56.6822                                                      \\
            \midrule
            upload                  & \multirow{2}{*}{1M}                         &   19.2296                                         &   19.1455 &   19.3161                                                      \\                                                                 
            materialize             &                                             &   89.5864                                         &   89.0209 &   90.1313                                                      \\
            \midrule
            upload                  & \multirow{2}{*}{6M}                         &  211.6758                                         &  211.0694 &  212.2839                                                      \\                                                                 
            materialize             &                                             &  270.3233                                         &  269.5967 &  270.9895                                                      \\
            \midrule
            upload                  & \multirow{2}{*}{6M}                         & 2068.5260                                         & 2060.3358 & 2077.2837                                                      \\                                                                 
            materialize             &                                             &  277.6001                                         &  276.0065 &  279.1456                                                      \\
            \bottomrule
        \end{tabular}
    \end{minipage}
    \begin{minipage}[b]{\textwidth}
        \centering
        \includegraphics[width=\textwidth]{figures/99-appendix/results-diagrams/write/hudi_upload_materialize/hudi_virtualiz_4_core.png}
        \caption{Histogram in log-scale displaying the contributions to the write latency of the upload and materialization steps in the legacy pipeline performed with 4 \glstext{CPU} cores}
        \label{fig:appx_hudi_virtualiz_breakdown_4_core}
    \end{minipage}
\end{figure}

\begin{figure}
    \centering
    \begin{minipage}[b]{\textwidth}
        \centering
        \captionof{table}{Contributions to the write latency of the upload and materialization steps in the legacy pipeline performed with 8 \glstext{CPU} cores}
        \label{tbl:appx_hudi_virtualiz_breakdown_8_cores}
        \begin{tabular}{l r S[table-format=5.4] S[table-format=5.4] S[table-format=5.4] S[table-format=5.4]} 
            \toprule
            {\multirow{2}{*}{Step\Tstrut\Bstrut}} & \multirow{2}{*}{{\thead{Number\\ of rows}}} & {\multirow{2}{*}{{\thead{Latency \\ (seconds)}}}} & \multicolumn{2}{c}{{\thead{Latency (seconds) \\95\% Confidence Interval}}}\\
                                    &                                             &                                                   & {low} & {high}                                                            \\
            \midrule
            upload                  & \multirow{2}{*}{10K}                        &    2.3815                                         &    2.3304 &    2.4358                                                      \\ 
            materialize             &                                             &   48.8485                                         &   48.1979 &   49.4467                                                      \\
            \midrule
            upload                  & \multirow{2}{*}{100K}                       &    3.4392                                         &    3.4081 &    3.4700                                                      \\                                                                 
            materialize             &                                             &   56.8428                                         &   56.3177 &   57.3685                                                      \\
            \midrule
            upload                  & \multirow{2}{*}{1M}                         &   18.8642                                         &   18.7808 &   18.9532                                                      \\                                                                 
            materialize             &                                             &   90.5153                                         &   90.0306 &   90.9718                                                      \\
            \midrule
            upload                  & \multirow{2}{*}{6M}                         &  207.6646                                         &  207.1606 &  208.2090                                                      \\                                                                 
            materialize             &                                             &  268.2752                                         &  267.3569 &  269.2456                                                      \\
            \midrule
            upload                  & \multirow{2}{*}{6M}                         & 2049.1371                                         & 2043.5991 & 2055.4782                                                      \\                                                                 
            materialize             &                                             &  275.7636                                         &  274.2773 &  274.2773                                                      \\
            \bottomrule
        \end{tabular}
    \end{minipage}
    \begin{minipage}[b]{\textwidth}
        \centering
        \includegraphics[width=\textwidth]{figures/99-appendix/results-diagrams/write/hudi_upload_materialize/hudi_virtualiz_8_core.png}
        \caption{Histogram in log-scale displaying the contributions to the write latency of the upload and materialization steps in the legacy pipeline performed with 8 \glstext{CPU} cores}
        \label{fig:appx_hudi_virtualiz_breakdown_8_core}
    \end{minipage}
\end{figure}

\cleardoublepage

%% The following label is necessary for computing the last page number of the body of the report to include in the "For DIVA" information
\label{pg:lastPageofMainmatter}

\cleardoublepage
\begin{comment}
%%%%%%%%%%%%%%%%%%%%%%%%%%%%%%%%%%%%%%%%%%%%%%%%%%%%%%%%%%%%%%%%%%%%%%%%%%%%%%%%%%%%%%%%
%%                                 FOR DIVA MATERIAL
%%%%%%%%%%%%%%%%%%%%%%%%%%%%%%%%%%%%%%%%%%%%%%%%%%%%%%%%%%%%%%%%%%%%%%%%%%%%%%%%%%%%%%%%
\clearpage\thispagestyle{empty}\mbox{} % empty page with backcover on the other side
\kthbackcover
\fancyhead{}  % Do not use header on this extra page or pages
\section*{€€€€ For DIVA €€€€}
\lstset{numbers=none} %% remove any list line numbering
\divainfo{pg:lastPageofPreface}{pg:lastPageofMainmatter}

% If there is an acronyms.tex file,
% add it to the end of the For DIVA information
% so that it can be used with the abstracts
% Note that the option "nolol" stops it from being listed in the List of Listings

% The following bit of ugliness is because of the problems PDFLaTeX has handling a non-breaking hyphen
% unless it is converted to UTF-8 encoding.
% If you do not use such characters in your acronyms, this could be simplified.
\ifxeorlua
\IfFileExists{lib/acronyms.tex}{
\section*{acronyms.tex}
\lstinputlisting[language={[LaTeX]TeX}, nolol, basicstyle=\ttfamily\color{black},
commentstyle=\color{black}, backgroundcolor=\color{white}]{lib/acronyms.tex}
}
{}
\else
\IfFileExists{lib/acronyms-for-pdflatex.tex}{
\section*{acronyms.tex}
\lstinputlisting[language={[LaTeX]TeX}, nolol, basicstyle=\ttfamily\color{black},
commentstyle=\color{black}, backgroundcolor=\color{white}]{lib/acronyms-for-pdflatex.tex}
}
{}
\fi
\end{comment}
\end{document}

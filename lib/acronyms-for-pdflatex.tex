%%% Local Variables:
%%% mode: latex
%%% TeX-master: t
%%% End:
%% The following command is used with glossaries-extra
\setabbreviationstyle[acronym]{long-short}
%% The form of the entries in this file is \newacronym{label}{acronym}{phrase}
%%                                      or \newacronym[options]{label}{acronym}{phrase}
%% see "User Manual for glossaries.sty" for the  details about the options, one example is shown below
%% note the specification of the long form plural in the line below
%%\newacronym[longplural={Debugging Information Entities}]{DIE}{DIE}{Debugging Information Entity}
%%
%% The following example also uses options
%%\newacronym[shortplural={OSes}, firstplural={operating systems (OSes)}]{OS}{OS}{operating system}
%%
%% note the use of a non-breaking dash in long text for the following acronym
%%\newacronym{IQL}{IQL}{Independent Q^^e2^^80^^91Learning}
%%
%% Notes
%% 1. you can't use the \gls() command in a heading - but you can get the short (\glsentryshort) 
%% or long version (\glsentryshort) or \glsentrylong or even the text entry (\glsentrytext) and then there is no problem

\newacronym{KTH}{KTH}{KTH Royal Institute of Technology}
\newacronym{ACID}{ACID}{Atomicity, Consistency, Isolation and Durability}
\newacronym{AI}{AI}{Artificial Intelligence}
\newacronym{ML}{ML}{Machine Learning}
\newacronym{BI}{BI}{Business Intelligence}
\newacronym[shortplural={RDDs}, firstplural={Resilient Distributed Datasets (RDDs)}]{RDD}{RDD}{Resilient Distributed Dataset}
\newacronym{OLAP}{OLAP}{On-Line Analytical Processing}
\newacronym{ELT}{ELT}{Extract Load Transform}
\newacronym{ETL}{ETL}{Extract Transform Load}
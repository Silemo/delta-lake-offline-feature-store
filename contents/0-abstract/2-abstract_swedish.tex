\markboth{\abstractname}{}
\begin{scontents}[store-env=lang]
swe
\end{scontents}
%%\warningExpl{Inside the following scontents environment, you cannot use a \textbackslash include{filename} as it will not end up in the for diva information. Additionally, you should not use a straight double quote character in the abstracts or keywords, use two single quote characters instead.}
\begin{scontents}[store-env=abstracts,print-env=true]
%%\generalExpl{Enter your Swedish abstract or summary here!}
%%\sweExpl{Alla avhandlingar vid KTH \textbf{måste ha} ett abstrakt på både \textit{engelska} och \textit{svenska}.\\
%%Om du skriver din avhandling på svenska ska detta göras först (och placera det som det första abstraktet) - och du bör revidera det vid behov.}
Behovet av att bygga modeller för maskininlärning (ML) baserade på ständigt ökande datamängder innebar nya utmaningar för datahanteringssystemen. Feature Stores har visat sig vara en effektiv lösning för att möjliggöra återanvändning av funktioner samtidigt som man organiserar datatransformationer och säkerställer konsekvens mellan funktionsteknik, modellutbildning och modellinferens. Nya publikationer visar att Hopsworks Feature Store uppvisar överlägsna prestandamätvärden i både tränings- och online-inferensfrågearbetsbelastningar jämfört med befintliga molnbaserade alternativ. I det här systemet är latensen för att utföra en skrivoperation minst en eller flera minuter, även för små datamängder (1 GB eller mindre). Denna begränsning tros vara en begränsning som är specifik för Spark, som systemet använder för att skriva data på offline feature store. Denna hypotes bekräftades redan i fallet med läslatens, där valet av ett Spark-alternativ, nämligen en Arrow Flight- och DuckDB-server, förbättrade prestandan avsevärt. Ett lovande tillvägagångssätt verkar vara att anta en ny datahanteringslösning, nämligen Delta Lake, och komma åt den med hjälp av ett Rust-bibliotek som heter delta-rs. Den här avhandlingen undersöker möjligheten att minska läs- och skrivfördröjningen i offline Feature Store genom att utöka delta-rs-biblioteket till att stödja Hopsworks Feature Store-filsystemet HopsFS och jämföra prestandan hos det gamla och det nyligen implementerade systemet. Efter den första iterativa systemimplementeringsfasen, som baserades på fasta krav, utvärderades systemet genom att utföra och mäta läs- och skrivoperationer i fyra olika CPU-konfigurationer, där antalet CPU-kärnor ökades till åtta. Experimenten utfördes femtio gånger för att uppskatta ett konfidensintervall som möjliggjorde en jämförande utvärdering av systemen. Resultaten bekräftade delta-rs-bibliotekets överlägsna prestanda jämfört med Spark-systemet i alla skrivoperationer med en tiofaldig minskning av latensen. Delta-rs överträffade också Spark-alternativet i läsoperationer med en tiofaldig minskning av latensen i alla utom experimentet med den största tabellen (60 miljoner rader), där förbättringen är av en mindre faktor. Dessa resultat uppmuntrar till framtida forskning som undersöker Spark-alternativet vid optimering av prestanda i småskaliga (1 GB - 100 GB) datahanteringssystem.
%%\engExpl{If you are writing your thesis in English, you can leave this until the draft version that goes to your opponent for the written opposition. In this way, you can provide the English and Swedish abstract/summary information that can be used in the announcement for your oral presentation.\\If you are writing your thesis in English, then this section can be a summary targeted at a more general reader. However, if you are writing your thesis in Swedish, then the reverse is true – your abstract should be for your target audience, while an English summary can be written targeted at a more general audience.\\This means that the English abstract and Swedish sammnfattning  
%%or Swedish abstract and English summary need not be literal translations of each other.}

%%\warningExpl{Do not use the \textbackslash glspl\{\} command in an abstract that is not in English, as my programs do not know how to generate plurals in other languages. Instead, you will need to spell these terms out or give the proper plural form. In fact, it is a good idea not to use the glossary commands at all in an abstract/summary in a language other than the language used in the \texttt{acronyms.tex file} - since the glossary package does \textbf{not} support use of more than one language.}

%%\engExpl{The abstract in the language used for the thesis should be the first abstract, while the Summary/Sammanfattning in the other language can follow}
\end{scontents}
\subsection*{Nyckelord}
\begin{scontents}[store-env=keywords,print-env=true]
% SwedishKeywords were set earlier, hence we can use alternative 2
\InsertKeywords{swedish}
%Första nyckelordet, Andra nyckelordet, Tredje nyckelordet, Fjärde nyckelordet
\end{scontents}
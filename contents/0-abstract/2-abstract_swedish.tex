\markboth{\abstractname}{}
\begin{scontents}[store-env=lang]
swe
\end{scontents}
%%\warningExpl{Inside the following scontents environment, you cannot use a \textbackslash include{filename} as it will not end up in the for diva information. Additionally, you should not use a straight double quote character in the abstracts or keywords, use two single quote characters instead.}
\begin{scontents}[store-env=abstracts,print-env=true]
%%\generalExpl{Enter your Swedish abstract or summary here!}
%%\sweExpl{Alla avhandlingar vid KTH \textbf{måste ha} ett abstrakt på både \textit{engelska} och \textit{svenska}.\\
%%Om du skriver din avhandling på svenska ska detta göras först (och placera det som det första abstraktet) - och du bör revidera det vid behov.}
Behovet av att bygga modeller för maskininlärning (ML) baserat på ständigt ökande datamängder har medfört nya utmaningar för datahanteringssystemen. Feature Stores har uppmärksammats som centraliserade dataplattformar som möjliggör återanvändning av features vid organisering av datatransformationer och säkerställer att feature-utveckling, modellträning och modellinferens är konsistenta. Nya publikationer visar att Hopsworks Feature Store har bättre prestanda än befintliga molnbaserade alternativ under modellträning och under beräkningslast för online-inferens. I sitt offline feature store lagrar Hopsworks feature store batchdata, dvs. historisk data, samlar det i feature groups, dvs. logiska tabeller med features, organiserade i Apache Hudi-tabeller, dvs. ett datahanteringslager, och lagrade i HopsFS, Hopsworks HDFS-distribution.Men även i detta system är latensen för att utföra en skrivoperation minst en eller flera minuter, även för små datamängder (1 GB eller mindre). En tidigare förbättring av läslatens med hjälp av en Arrow Flight- och DuckDB-server tyder på att denna begränsning orsakas av Spark, som systemet använder för att skriva data i Apache Hudi-tabeller. En lovande metod för att undvika Spark verkar vara att använda Delta Lake istället för Apache Hudi för datahantering, och använda ett Rust-bibliotek som heter delta-rs för dataåtkomst. Den här avhandlingen undersöker möjligheten att minska läs- och skrivfördröjningen i offline-funktionslagret genom att utöka delta-rs-biblioteket för att stödja Hopsworks-funktionslagrets filsystem som kallas HopsFS och jämförande utvärdera prestanda för det äldre och det nyligen implementerade systemet. Två större iterationer av utveckling av stöd för HopsFS-lagring i delta-rs utfördes för att uppfylla de strikta krav på produktionsfärdighet som definierades före utvecklingen. Systemet utvärderades sedan genom att utföra och mäta läs- och skrivoperationer i fyra olika CPU-konfigurationer, där antalet CPU-kärnor ökades upp till åtta. Experimenten utfördes femtio gånger för att uppskatta ett konfidensintervall, vilket möjliggjorde en korrekt jämförande utvärdering av systemen.  Resultaten bekräftade delta-rs-bibliotekets överlägsna prestanda jämfört med Spark-systemet i alla skrivoperationer med en latensminskning från tio upp till fyrtio gånger. Delta-rs överträffade också Spark-alternativet i läsoperationer med en latensminskning på fyrtiosju procent, upp till fyrtio gånger.
Dessa resultat uppmuntrar till framtida forskning där Spark alternativet undersöks för att optimera prestanda i småskaliga (1 GB - 100 GB) datahanteringssystem. Det utvecklade systemet kommer att användas i produktionsmiljön för Hopsworks feature store.
%%\engExpl{If you are writing your thesis in English, you can leave this until the draft version that goes to your opponent for the written opposition. In this way, you can provide the English and Swedish abstract/summary information that can be used in the announcement for your oral presentation.\\If you are writing your thesis in English, then this section can be a summary targeted at a more general reader. However, if you are writing your thesis in Swedish, then the reverse is true – your abstract should be for your target audience, while an English summary can be written targeted at a more general audience.\\This means that the English abstract and Swedish sammnfattning  
%%or Swedish abstract and English summary need not be literal translations of each other.}

%%\warningExpl{Do not use the \textbackslash glspl\{\} command in an abstract that is not in English, as my programs do not know how to generate plurals in other languages. Instead, you will need to spell these terms out or give the proper plural form. In fact, it is a good idea not to use the glossary commands at all in an abstract/summary in a language other than the language used in the \texttt{acronyms.tex file} - since the glossary package does \textbf{not} support use of more than one language.}

%%\engExpl{The abstract in the language used for the thesis should be the first abstract, while the Summary/Sammanfattning in the other language can follow}
\end{scontents}
\subsection*{Nyckelord}
\begin{scontents}[store-env=keywords,print-env=true]
% SwedishKeywords were set earlier, hence we can use alternative 2
\InsertKeywords{swedish}
%Första nyckelordet, Andra nyckelordet, Tredje nyckelordet, Fjärde nyckelordet
\end{scontents}
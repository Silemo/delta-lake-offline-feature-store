\markboth{\abstractname}{}
\begin{scontents}[store-env=lang]
ita
\end{scontents}
\begin{scontents}[store-env=abstracts,print-env=true]
La necessità di costruire modelli di \textit{Machine Learning (ML)} basati su quantità sempre maggiori di dati ha posto nuove sfide ai sistemi di gestione dei dati. I \textit{feature stores} sono emersi come una piattaforma di dati centralizzata che consente il riutilizzo delle \textit{features}, organizzando al contempo le trasformazioni dei dati e garantendo la coerenza tra il \textit{feature engineering}, il \textit{training} e l'\textit{inference} dei modelli. Recenti pubblicazioni dimostrano che il \textit{feature store} di Hopsworks supera le alternative esistenti basate su \textit{cloud} nelle \textit{query} di \textit{training} e di \textit{online inference}. Nel suo \textit{feature store offline}, Hopsworks memorizza dati \textit{batch} o storici, raccogliendoli in \textit{feature groups}, ossia tabelle logiche di \textit{features}, organizzate in Apache Hudi \textit{tables}, ossia un livello di gestione dei dati, e memorizzate su HopsFS, la versione di HDFS di Hopsworks. Tuttavia, anche in questo sistema, la latenza per eseguire un'operazione di scrittura è di almeno uno o più minuti, anche per piccole quantità di dati (1 GB o meno). Un precedente miglioramento della latenza di lettura utilizzando un server Arrow Flight e DuckDB suggerisce che questa limitazione è causata da Spark, che il sistema utilizza per scrivere i dati sulle Apache Hudi \textit{tables}. 
Un approccio promettente per evitare l'uso di Spark sembra essere l'adozione di Delta Lake al posto di Apache Hudi per gestire e accedere ai dati utilizzando una libreria Rust chiamata delta-rs. Questa tesi studia la possibilità di ridurre la latenza di lettura e scrittura nel \textit{offline feature store} di Hopsworks espandendo la libreria delta-rs per supportare il file system del \textit{feature store} di Hopsworks chiamato HopsFS e valutando in modo comparativo le prestazioni del sistema \textit{legacy} e di quello appena implementato. Sono state sviluppate due iterazioni principali di supporto allo storage in delta-rs per HopsFS, per soddisfare i severi requisiti di produzione definiti prima dello sviluppo. Il sistema è stato quindi valutato eseguendo e misurando le operazioni di lettura e scrittura in quattro diverse configurazioni di CPU, aumentando il numero di \textit{core} della CPU fino a otto. Gli esperimenti sono stati eseguiti cinquanta volte per stimare un intervallo di confidenza, consentendo un'accurata valutazione comparativa dei sistemi. I risultati hanno confermato la superiorità della libreria delta-rs rispetto al sistema Spark in tutte le operazioni di scrittura, con una riduzione della latenza da dieci a quaranta volte. Delta-rs ha superato l'alternativa Spark anche nelle operazioni di lettura, con una riduzione della latenza dal quarantasette percento, fino a quaranta volte. Questi risultati incoraggiano la ricerca futura sull'alternative a Spark per l'ottimizzazione delle prestazioni nei sistemi di gestione dei dati su piccola scala (1 GB - 100 GB). Il sistema sviluppato troverà applicazione nell'ambiente di produzione del \textit{feature store} Hopsworks.
\end{scontents}
\subsection*{Parole chiave}
\begin{scontents}[store-env=keywords,print-env=true]
Machine Learning, Feature Store, Spark, Delta Lake, libreria delta-rs, Latenza di lettura/scrittura
\end{scontents}
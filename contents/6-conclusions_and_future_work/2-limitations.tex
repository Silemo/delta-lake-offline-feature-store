The limitations of this study mostly derive from the constraints of resources, in terms of time and computational resources, and scope, which is mostly linked to the defined use case (Section \ref{subsec:use_case}). 

Being the scope of this project improving the latency of the Hopsworks Offline Feature Store, this is the technology to which the implementation and the experiments were based on. This outlines the limited generalization of the results obtained, which are biased from the use of a technology in collaboration with the company developing it. Additionally, a specific use case was defined (Section \ref{subsec:use_case}) to choose the \gls{CPU} loads and data loads on which the experiments would be conducted. This helps to define a perimeter of the thesis contribution, but also limits the thesis impact to this specific use case, requiring more research to verify the same hypothesis in a difference scenario.

On the computational resources provided, while great in size, they were used on a shared environment that could only be used for a limited amount of time and only if it was not operating other, more critical workloads. Time also played a role in limiting the number of experiments to be conducted as to calculate a 95\% confidence interval, all experiments were run 50 times, which increases by a great factor the time required to perform all experiments. 
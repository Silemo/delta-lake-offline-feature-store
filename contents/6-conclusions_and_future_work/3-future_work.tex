The results and limitations of this thesis offer a good starting point for future work. As outlined in the limitations section, this thesis's scope and resources were limited. Conducting new experiments on the system performance by relaxing one or more constraints could bring new results that can be more general.

Considering the system research contribution of this thesis, this could be expanded mainly for Hopsworks \gls{AB} needs. The code allowing delta-rs to read and write Delta Tables on \gls{HopsFS} will probably see use mainly in Hopsworks \gls{AB}, as even if \gls{HopsFS} is an open-source technology, it does not see much use outside the Hopsworks tools. Nonetheless, future system contributions could still use the code developed in this thesis as a baseline to be compared with new delta-rs implementations or other ways to read and write on an offline feature store.

Future work on the system evaluation could expand on one or more of these aspects: data, pipelines, and experimental environment. Expanding on data would mean running the experiments with larger tables (600M rows, 6BN rows, etc.) to explore and verify if a threshold where a Spark-based system performs better than delta-rs is present and at which table size. Also, making variations on the data sources would be a valid approach, although this would make this thesis an invalid baseline. The defined pipelines are specific to delta-rs or the Hopsworks architecture. Verifying the speed of other offline feature stores as Databricks would help generalize the results in a broader area. This would help determine which approach—Spark or delta-rs—performs best across different systems.
As mentioned in the limitations, the experimental environment was a shared environment with extensive resources, but the current machine usage by other company employees varied. Using clean, isolated hardware could help future research verify this thesis's findings, isolating the variable results of a shared environment.
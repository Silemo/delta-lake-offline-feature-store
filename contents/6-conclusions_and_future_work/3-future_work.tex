The results and limitations of this thesis offer a good starting point for future work. As outlined in the limitations section before, this thesis's scope and resources were limited so conducting new experiments on the system performance by relaxing one or more constraints, could bring new results that can be more general.

Considering the system research contribution of this thesis, this could be expanded mostly for Hopsworks \gls{AB} needs as the code works with key components of their infrastructure, e.g. \gls{HopsFS}, that while open-source does not see much use outside the company. The code is unlikely to see use in the delta-rs library, as even if this is a solid contribution, it only fits a single company's use case, and it is very limited in their applications outside it. Future system contributions could still use the code developed in this thesis as a baseline, to be compared with new delta-rs implementations or other ways to read and write on an Offline Feature Store.

The future work that could be carried out on the system evaluation could expand on one or more of these aspects: data, pipelines, and experimental environment. Expanding on data would mean running the experiments with larger tables (600M rows, 6BN rows, etc.) to explore and verify if a threshold where a Spark-based system performs better than delta-rs is present, and at which table size. Also making variations on the data sources would be a valid approach, although this would make this thesis an invalid baseline. The pipelines that were defined are specific to delta-rs or the Hopsworks architecture. Verifying the speed of other Offline Feature Stores as Databricks' would help generalize the results on a broader area. This would help clear out which approach performs best (between Spark and delta-rs) across different systems.
The experimental environment as said in the limitations, was a shared environment with large resources, but varying on the current machine usage by other company's employees. Using clean isolated hardware could help future research verify this thesis's findings, isolating the variable results of a shared environment.
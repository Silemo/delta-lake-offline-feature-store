The Hopsworks feature store \cite{10.1145/3626246.3653389} first used Apache Hudi for their offline feature store, as it was the first open-sourced data lakehouse in 2017. Recently, Hopsworks \gls{AB} added support for using Delta Lake as an offline feature store, following its clients' requests. Spark is a query engine in the system, i.e., it executes the query (read, write, or delete) on the offline feature store. Running the system showed that even a write operation on a small dataset, consisting of 1 GB of data or less, takes one or more minutes to complete.

This hurts Hopsworks' typical use case, which sits between tests on small quantities of data (1 GB - 10 GB) and production scenarios on a larger scale but still relatively small (10 GB - 100 GB).

This research's underlying hypothesis is that this slow transaction time is a Spark-specific issue. This has led Hopsworks to adopt Spark alternatives~\cite{Khazanchi1801362} for reading in their Apache Hudi system. Delta Lake supports Spark alternatives for accessing and querying the data, and of particular interest is the delta-rs library~\footnote{Project repository available at \url{https://github.com/delta-io/delta-rs}} that enables Python access to Delta Lake tables without using Spark. 
However, the delta-rs library does not support \gls{HDFS}, and consequently \gls{HopsFS}~\cite{niaziHopsFSScalingHierarchical2017}.

% Research Question
\subsection{Research Questions}
\label{subsec:researchQuestion}
This research project has the ultimate objective to evaluate and compare the performance of the current Spark system that operates on Apache Hudi to a Rust system that uses delta-rs library~\footnotemark[\value{footnote}] operates on Delta Lake, using \gls{HopsFS}~\cite{niaziHopsFSScalingHierarchical2017}. To achieve this, support for \gls{HDFS} (and thus also \gls{HopsFS}) must be added to the delta-rs library so that it can be compatible with the Hopsworks system. Therefore, the project addresses the following two \glspl{RQ}:
\begin{enumerate}
    \item[RQ1:] How can we add support for \gls{HDFS} and \gls{HopsFS} to the delta-rs library to enable reading and writing on Delta Lake tables on the Hopsworks offline feature store?
    \item[RQ2:] What is the difference in read and write latency and throughput between the current legacy system operating on the Hopsworks offline feature store and the delta-rs library operating on HopsFS?
\end{enumerate}
Note that RQ2 was formulated to reflect the experiments that will be performed. This is because, even though measured performance should be similar if delta-rs will be included in the Hopsworks client in the future, it is formally not operating on the offline feature store but only using the same file system, \gls{HopsFS}.

%Some text
% Small summary of the problem and formulation of a single research question

% \subsection{Scientific and engineering issues}
% % Outline of the various problems to address during implementation
% Delta-rs \cite{DeltaioDeltars2024}, as the name suggests, is a Rust \cite{RustProgrammingLanguage} library, that offers Python bindings. Rust is a compiled language, so it does not need an interpreter like Python or a virtual environment like Java. This means it is straightforward to embed and use Rust code as a library in another language, such as Python. 

% Currently, delta-rs does not support \gls{HDFS} and therefore, \gls{HopsFS}\cite{niaziHopsFSScalingHierarchical2017}. This means that adding \gls{HDFS} support for delta-rs becomes a requirement of this project. Additionally, it should be noted that to match the dependencies used in the repository, the object\_store \cite{Object_storeRust} interface of Apache DataFusion \cite{ApacheDataFusionApache} should be used.
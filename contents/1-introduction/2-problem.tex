The Hopsworks Feature Store \cite{HopsworksBatchRealtime2024} first used Apache Hudi for their Offline Feature Store, as it was the first open-sourced Lakehouse in 2017. Recently, Hopsworks \gls{AB} added support for using Delta Lake as Offline Feature Store, following its clients' requests. In the system Spark is used as a query engine, i.e. executes the query (read, write, or delete) on the Offline Feature Store. Running the system revealed that even a write operation on a small of data (only 1 GB of data or less) takes one or more minutes to complete.

This hurts Hopsworks' typical use-case which sits between tests on small quantities of data (scale between 1-10 GBs) and production scenarios on a larger scale, but still relatively small (scale between 10-100 GBs).

The research hypothesis that guides this research is that this slow transaction time is a Spark-specific issue. This is why Hopsworks has already adopted Spark alternatives \cite{Khazanchi1801362} for reading in their Apache Hudi system. Delta Lake supports Spark alternatives for accessing and querying the data, and of particular interest is the delta-rs library \cite{DeltaioDeltars2024} that enables Python access to Delta Lake tables, without having to use Spark. 
However, the delta-rs \cite{DeltaioDeltars2024} does not support \gls{HDFS}, and consequently \gls{HopsFS} \cite{niaziHopsFSScalingHierarchical2017}.

% Research Question
\subsection{Research Question}
\label{subsec:researchQuestion}
This research project has the ultimate objective to evaluate and compare the performance of the current Spark system that operates on Apache Hudi, to a Rust system that uses delta-rs library \cite{DeltaioDeltars2024} operates on Delta Lake, using \gls{HopsFS} \cite{niaziHopsFSScalingHierarchical2017}. To achieve this, support for \gls{HDFS} (and thus also \gls{HopsFS}) must be added to the delta-rs library \cite{DeltaioDeltars2024}, so that it can be compatible with the Hopsworks system. Thus the project addresses the following two \glspl{RQ}:
\begin{enumerate}
    \item[RQ1:] How can we add support for \gls{HDFS} and \gls{HopsFS} to the delta-rs library?
    \item[RQ2:] What is the difference in latency and throughput between the current legacy system (Spark-based in writing) reading and writing to Apache Hudi compared to a delta-rs library-based reading and writing to Delta Lake, in \gls{HopsFS}?
\end{enumerate}

%Some text
% Small summary of the problem and formulation of a single research question

% \subsection{Scientific and engineering issues}
% % Outline of the various problems to address during implementation
% Delta-rs \cite{DeltaioDeltars2024}, as the name suggests, is a Rust \cite{RustProgrammingLanguage} library, that offers Python bindings. Rust is a compiled language, and as such it does not need an interpreter like Python or a virtual environment like Java. This means that it is particularly easy to embed and use Rust code as a library in another language such as Python. 

% Currently, delta-rs does not support \gls{HDFS} and therefore, \gls{HopsFS}\cite{niaziHopsFSScalingHierarchical2017}. This means that adding \gls{HDFS} support for delta-rs becomes a requirement of this project. Additionally, it should be noted that to match the dependencies used in the repository, the object\_store \cite{Object_storeRust} interface of Apache DataFusion \cite{ApacheDataFusionApache} should be used.
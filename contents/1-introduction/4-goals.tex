%%\sweExpl{Mål}
%%\sweExpl{Skilj på syfte och mål. Syftet är att åstakomma en förändring i något. Målen är vad som konkret skall göras för att om möjligt uppnå den önskade förändringen (syfte). }
%%
%%\generalExpl{State the goal/goals of this degree project.}
The accomplishment of the project's purpose (namely, reducing the latency (seconds) and thus increasing the data throughput (rows/second) for reading and writing on Delta Lake tables on \gls{HopsFS}) is bound to a list of \glspl{G}, here set. These are also related to the set of \glspl{RQ}, outlining a clear structure of the various project milestones.

\begin{enumerate}
    \item \glspl{G} aimed to answer RQ1: 
        \begin{enumerate}
            \item[G1:] Understand delta-rs library architecture and dependencies.
            \item[G2:] Identify what needs to be implemented to add \gls{HDFS} support to the delta-rs library.  
            \item[G3:] Implement \gls{HDFS} support in the delta-rs library.
            \item[G4:] Verify that \gls{HDFS} support also works for \gls{HopsFS}.
        \end{enumerate}
    \item \glspl{G} aimed to answer RQ2:
        \begin{enumerate}
            \item[G5:] Design the experiments to be conducted to evaluate the difference in performance between the current legacy access (Spark-based in writing) to Apache Hudi compared a the delta-rs library-based access to Delta Lake, in \gls{HopsFS}. 
            \item[G6:] Perform the designed experiments.
            \item[G7:] Visualize the experiments' results, focusing on allowing an effective comparison of performances.
            \item[G8:] Analyze and interpret the results in a dedicated thesis report section.
        \end{enumerate}
\end{enumerate}

%%\generalExpl{In addition to presenting the goal(s), you might also state what the deliverables and results of the project are.}
Associated with these \glspl{G} several \glspl{D} will be created. 
\begin{enumerate}
    \item[D1:] Code implementation adding support to \gls{HDFS} and \gls{HopsFS} in the delta-rs library. This \gls{D} is related to the completion of goals G1--G4. This deliverable also represents the system implementation contribution of the project.
    \item[D2:] Experiment results on the difference in performance between current legacy access (Spark-based in writing) to Apache Hudi compared a the delta-rs library-based access to Delta Lake, in \gls{HopsFS}.
    This \gls{D} is related to the completion of goals G5--G7.
    \item[D3:] This thesis document, provides more detail on the implementation, design decisions, expected performance, and analysis of the results.
    This \gls{D} is a comprehensive report of all the thesis work, also including the analysis of results defined in G8.
\end{enumerate}
\sweExpl{svensk: Introduktion}

\sweExpl{Ofta kommer problemet och problemägaren från industrin där man önskar en specifik lösning på ett specifikt problem. Detta är ofta ”för smalt” definierat och ger ofta en ”för smal” lösning för att resultatet skall vara intressant ur ett mer allmänt ingenjörsperspektiv och med ”nya” erfarenheter som resultat. Fundera tillsammans med projektets intressenter (student, problemägare och akademi) hur man skulle kunna använda det aktuella problemet/förslaget för att undersöka någon ingenjörsaspekt och vars resultat kan ge ny eller kompletterande erfarenhet till ingenjörssamfundet och vetenskapen.\\slöser man en del eller hela delen av det ursprungliga problemet.\\Erfarenheten kommer ur en frågeställning som man i examensarbetet försöker besvara med tidigare och andras erfarenhet, egna eller modifierade metoder som ger ett resultat vilket kan användas för att diskutera ett svar på undersökningsfrågan.\\Detta stycke skall alltså, förutom det ursprungliga ”smala” problemet, innehålla  vad som skall undersökas för att skapa ny ingenjörserfarenhet och/eller vetenskap.}

\engExpl{The first paragraph after a heading is not indented, all of the
  subsequent paragraphs have their first line indented.}
  
This chapter describes the specific problem that this thesis addresses, the context of the problem, the
goals of this thesis project, and outlines the structure of the thesis.\\

\generalExpl{Give a general introduction to the area. (Remember to use appropriate references in this and all other sections.)}

% One can use either biblatex or bibtex - set as the option for the document at the top of this file
\ifbiblatex
\engExpl{We use the \emph{biblatex} package to handle our references.  We
use the command \texttt{parencite} to get a reference in parenthesis, like
this \textbackslash parencite\{heisenberg2015\} resulting in \parencite{heisenberg2015}.  It is also possible to include the author as part of the sentence using \texttt{textcite}, like talking about the work of \textbackslash textcite\{einstein2016\} resulting in \textcite{einstein2016}.\\
This also means that you have to change the include files to include biblatex and change the way that the \texttt{reference.bib} file is included.}
\else
\engExpl{We use the \emph{bibtex} package to handle our references.  We, therefore,
use the command \textbackslash cite\{farshin\_make\_2019\}. For example, Farshin, \etal described how to improve LLC
cache performance in \cite{farshin_make_2019} in the context of links running
at \qty{200}{Gbps}.}
\fi

\engExpl{Use the glossaries package to help yourself and your readers.
Add the acronyms and abbreviations to lib/acronyms.tex. Some examples are shown below:}
In this thesis, we will examine the use of \glspl{LAN}. In this thesis, we will
assume that \glspl{LAN} include \glspl{WLAN}, such as \gls{WiFi}.
%%\sweExpl{Undersökningsmetod}
%%sweExpl{Här anger du vilken vilken övergripande undersökningsstrategi eller metod du skall använda för att försöka besvara den akademiska frågeställning och samtidigt lösa det e v ursprungliga problemet. Ofta kan man använda ”lösandet av ursprungsproblemet” som en fallstudie kring en akademisk frågeställning. Du undersöker någon intressant fråga i ”skarpt” läge och samlar resultat och erfarenhet ur detta.\\
%%Tänk på att företaget ibland måste stå tillbaka i sin önskan och förväntan på projektets resultat till förmån för ny eller kompletterande ingenjörserfarenhet och vetenskap (ditt examensarbete). Det är du som student som bestämmer och löser fördelningen mellan dessa två intressen men se till att alla är informerade. }
%%\generalExpl{Introduce your choice of methodology/methodologies and method/methods – and the reason why you chose them. Contrast them with and explain why you did not choose other methodologies or methods. (The details of the actual methodology and method you have chosen will be given in Chapter~\ref{ch:method}. Note that in Chapter~\ref{ch:method}, the focus could be research strategies, data collection, data analysis, and quality assurance.)\\}
This work starts from a few \glspl{IN}, provided by Hopsworks, and a few \glspl{PA} validated through a literature study. \\ Hopsworks's \glspl{IN} are:
\begin{itemize}
    \item[IN1 :] the Hopsworks Feature Store using the legacy pipeline (Spark-base in writing) has a high latency (seconds) and low throughput (rows/second) in reading and writing operations when on a "small scale" (1 GB - 100 GB). This highlights the potential for using Spark alternatives in the "small scale" use case.
    \item[IN2 :] Hopsworks, adapting to their customer needs, supports the Delta Lake table format. Improving the speed of read and write operations on this table format, would improve a typical use case for Hopsworks Feature Store users.
\end{itemize}
\glspl{PA} are... These assumptions will be validated in Chapter \ref{ch:background}. 
\begin{itemize}
    \item[PA1 :] Python is the most popular programming language and the most used in data science workflows. \gls{ML} and \gls{AI} developers prefer Python tools to work. This means that Python libraries with high performance will typically be preferred over alternatives (even more efficient) that are \gls{JVM} or other environments based.
    \item[PA2 :] Rust libraries have proven to have the chance to improve performance over C/C++ counterparts (Polars over Pandas). A Rust implementation could strongly improve reading and writing operations on the Hopsworks Feature Store.
\end{itemize}

The project aims at fulfilling the \glspl{IN} with a system implementation approach. First, a \gls{HDFS} storage support will be written for the delta-rs library to extend the Rust library support to \gls{HopsFS} \cite{niaziHopsFSScalingHierarchical2017}. Then, an evaluation structure will be designed and used to compare the performances of the current legacy (Spark-based in writing) system and the new Rust-based pipeline. The two approaches will be tested with datasets of different sizes (between 1 GB and 100 GB). This is critical to identify if the same tool should be used for all scenarios or if they perform differently. The critical metrics that will be used to evaluate the system are read and write operations data throughout (the higher, the better) measured in rows per second. These were chosen as they most affect the computation time of pipelines accessing Delta Lake tables.

\subsection{Delimitations}
    \label{subsec:delimitations}
    %In this section I will describe the boundaries/limits of your thesis project and what you are explicitly not going to do. This will help you bind your efforts - as you have clearly defined what is out of the scope of this thesis project. Explain the delimitations. These are all the things that could affect the study if they were examined and included in the degree project.
The project is conducted in collaboration with Hopsworks AB, and as such the implementation will focus on working with their system using \gls{HopsFS}. While the consideration drawn from these results cannot be generalized and be true for any system, they can still provide an insight into Apache Spark limitations, and on which tools perform better in different use cases. 